\mode<article>
Cuando un equipo profesional se disponoe a resolver 
un problema Ingenieril o físico, es posible dividir la 
tarea en tres etapas, como se puestra en la Figura 
\ref{FiguraProbIngFis}.  
El \emph{preproceso} 
abarca toda la etapa de planteo del problema
desde la identificación del modelo físico a 
usar, las condiciones de contorno, toma de 
medidas, modelo matemático, etc. 

Como \emph{proceso} del problema, podemos ubicar
la etapa de implementación del modelo
matemático. Cuando se usen computadoras
para resolver esta parte del problema,
la discreitzación y matricialización de 
las funciones de estado y de las variables
independientes pueden ubicarse en esta 
etapa. La resolución, i.e. la ejecucion
del comando final , y el registro de los
resultados pueden ubicarse aquí.

Por último, en la etapa de 
\emph{postproceso} podemos ubicar todas
las tareas de medición y análisis sobre 
los resultados obtenidos en la etapa
anterior. 

 \begin{figure}
  \includeslide[width=\textwidth]{FrameProbIngFis}
  \caption{División del problema a resolver en etapas de análisis \label{FiguraProbIngFis}}

\end{figure}


\mode*

\begin{frame}<presentation>[label=FrameProbIngFis]{Problema Ingenieril o Físico}
\begin{columns}[t]
\column{0.3\textwidth}
\begin{beamercolorbox}[ht=1cm,sep=10pt]{header1}
\centering Preproceso \par
\end{beamercolorbox}
\begin{itemize}
\item “dibujo” del problema
\item recinto de validez

\item Modelo Físico\tikz\node[overlay](n1){};
\item Condiciones de contorno\tikz\node[overlay](n2){};
\end{itemize}
\column{0.3\textwidth}
\begin{beamercolorbox}[ht=1cm,sep=10pt]{header2}
\centering Proceso \par
\end{beamercolorbox}
\begin{itemize}
\item  \tikz\node[overlay](n3){}; Matricialización
\item Lectura de datos
\item Resolución
\item Escritura de resultados
\end{itemize}
\pause
\tikz[overlay] \draw[->,draw=red] (n1) -- (n3);
\tikz[overlay] \draw[->,draw=red] (n2) -- (n3);
\pause
\column{0.3\textwidth}
\begin{beamercolorbox}[ht=1cm,sep=10pt]{header3}
\centering Postproceso \par
\end{beamercolorbox}
\begin{itemize}
\item Mediciones Ingenieriles 
\item Información Gráfica
\item Interpretación de resultados
\end{itemize}
\end{columns}

\end{frame}

\mode<all>
