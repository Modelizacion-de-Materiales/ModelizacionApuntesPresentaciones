
\newcommand{\timebar}[1]{
\begin{tikzpicture}
\fill [gray] (0,0) rectangle (1.5,1); 
\fill [Cerulean] (1.5,0) rectangle (4,1); 
\draw (0,0) rectangle (4,1);
\node at (2,0.5) {\textcolor{white}{#1}};
\end{tikzpicture}
}

\mode<article>

En todas las etapas de la resolución 
del problema, el ingeniero o profesional 
hará uso de ciertas herramientas que 
le permitirán resolver el problema. 
La participación participación 
(i.e. tiempo de trabajo) estará repartida
entre el usuario y la herramientas
en sí (típicamente la pc 
o software a utilizar). 

\mode*

\begin{frame}<presentation>[fragile,label=FrameResulucionProblemas]{Resolucion de Problemas}

%\begin{columns}[t]
%\column{0.3\textwidth}
\timebar{\bfseries{  PREROCESO }}
%\column{0.3\textwidth}
\timebar{\bfseries{  PROCESO }}
%\column{0.3\textwidth}
\timebar{\bfseries{  POSPROCESO }}
%\end{columns}

\vspace{2cm} 

\centering
\tikz\fill [Cerulean] (0,0) rectangle (0.5,0.5) ;
Usuario, Ingeniero o Profesional

\tikz\fill [gray] (0,0) rectangle (0.5,0.5);
Herramienta de Cáclulo

\end{frame}

\mode<all>
