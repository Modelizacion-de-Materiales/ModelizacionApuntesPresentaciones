\subsubsection{Apertura y cierre de archivos}
\mode<article>

En todos los lenguajes de programación se utiliza algún comando
para abrir los archivos, por ejemplo \textbf{fopen} en matlab.
Este comando suele aceptar un argumento para indicar el
nombre de archivo a abrir, que puede ser una variable 
de tipo string o cadena de caracteres. Otro argumento
que se le da a \textbf{fopen} es el permiso con el 
que el archivo debe ser abierto, por ejemplo
de escritura (generalmente \textbf{w}, solo lectura \textbf{r}
, o para agregar contenido al final o \emph{append} \textbf{a}. 
En la  \autoref{FigIOAbrirArchivos} se esquematiza
la utilización de estos comandos. 

Es necesario notar que el comando \textbf{open} devuelve
una variable de salida que en el ejemplo se asigna 
en la variable \textbf{fid}. 
La importancia de la misma radica en que a partir
de la asignación \textbf{fid} se convierte en 
un \emph{indicador del archivo}, una especie de puntero
a la primer línea disponible del archivo. En caso
de haberlo abierto con permisos de escritura, 
el puntero indica la primer línea del 
archivo que debe escribirse. Por el contrario
al abrir el archivo con permisos de lectura
el puntero apunta a la primer
línea que puede leerse. El concepto quedará 
claro enseguida. 

\begin{figure}
\includeslide[width=\textwidth]{FrameIOAbrirArchivos}
\caption{Comandos básicos para la apertura de un archivo\label{FigIOAbrirArchivos}}
\end{figure}

\mode*

\begin{frame}<presentation>[label=FrameIOAbrirArchivos]
\frametitle{Apertura y cierre de archivos}

\begin{columns}[T]
\column{0.5\textwidth}
\hfill Abrir archivo existente para lectura:
\column{0.5\textwidth}
\begin{codeblock}
\verbatiminput{./CODEXAMPLES/open_file.m}
\end{codeblock}
\end{columns}

\begin{columns}[T]
\column{0.5\textwidth}
\hfill Abrir archivo \emph{inexistente} para lectura:
\column{0.5\textwidth}
\begin{codeblock}
\verbatiminput{./CODEXAMPLES/open_file_noex.m}
\end{codeblock}
\end{columns}

\begin{columns}[T]
\column{0.5\textwidth}
\hfill Abrir otro archivo  para \emph{escritura}:
\column{0.5\textwidth}
\begin{codeblock}
\verbatiminput{./CODEXAMPLES/open_otherfile.m}
\end{codeblock}
\end{columns}

\end{frame}

\mode<all>
