\subsubsection{Escritura y Lectura en Bloque}

\mode<article>

Cuando solo se quiere guardar una matriz o lista de datos de igual tipo (numeros reales, etc)
se puede usar la escritura en bloque como muestra la \autoref{FigWriteBlock} para \texttt{Matlab}. 

\begin{figure}
  \includeslide[width=\textwidth]{FrameWriteBlock}
  \caption{\protect\label{FigWriteBlock} Escritura y lectura de archivos de listas o matrices en 
  \texttt{Matlab}}
\end{figure}

\mode*


\begin{frame}<presentation>[label=FrameWriteBlock]
  \frametitle{Escribir y leer en Bloque}

  \begin{minipage}[t][2cm][t]{\textwidth}
    \begin{columns}[t]
      \column{0.5\textwidth}
      \begin{codeblock}
	\verbatiminput{CODEXAMPLES/writeblock.m}
      \end{codeblock}
      \column{0.5\textwidth}
      \tiny
	\VerbatimInput[lastline=10, fontsize=\tiny]{CODEXAMPLES/table.dat}
    \end{columns}
  \end{minipage}

    \begin{columns}
      \column{0.5\textwidth}
      \begin{codeblock}
	\verbatiminput{CODEXAMPLES/loadblock.m}
      \end{codeblock}
      \column{0.5\textwidth}
      \includegraphics[width=0.95\textwidth]{CODEXAMPLES/xcuad.png}
    \end{columns}
\end{frame}

\mode<all>

