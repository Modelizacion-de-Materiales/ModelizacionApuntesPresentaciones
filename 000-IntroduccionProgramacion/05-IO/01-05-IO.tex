\subsection{Entrada y Salida de datos}

\mode<article>

El registro de los resultados de un problema es tan 
importante como la trazabilidad de los datos 
que se tuvieron en cuenta para resolverlo. Por lo tanto, 
en general se guardan ambos tipos de datos en archivos, 
que serán escritos y leídos por nuestra herramienta 
informática oportunamente. notar entonces que la legibilidad
y la compatibilidad con otras herramientas 
serán características obligadas de los archivos
de entrada y salida generados.

La secuencia de comandos que deben usarse para estos
fines responden al diagrama de flujo de la figura
\ref{FigFlujoIO}. Siempre habrá un comando para 
abrir el archivo, habrá comandos para leer (o 
escribir) en forma secuencial  el archivo de entrada
(o de salida), y por último se ejecutará un comando 
para cerrar el archivo.

\begin{figure}
  \includeslide[width=\textwidth]{FrameEstructuraIO}
\caption{Flujo de ejecución para la secuencia de 
lectura/escritura de los archivos de registro de resultados
y de datos de entrada \label{FigFlujoIO}.}
\end{figure}

\mode*


\begin{frame}<presentation>[label=FrameEstructuraIO]
\frametitle{Input-Output / Entrada-Salida de Datos}
\tikzstyle{every node}=[draw, line width=2pt,
  outer sep=0.1cm, align=center,text height=8pt]
\tikzstyle{every path}=[line width=2pt, >=latex]
\begin{tikzpicture}
\node [ minimum width=\textwidth, fill=Purple, draw, 
line width=2pt, text=white] (title)  { Estructura IO};
\node [ minimum width=0.5\textwidth, fill=BrickRed, 
below=1cm of title, text=white ]
(abrir) {ABRIR};
\node [ text width=0.35\textwidth, fill=Khaki, 
below=0.5cm of abrir.south east ] (leer) 
{Leer en forma secuencial 
\\
\tiny \texttt{linea = read(archivo)}
};
\node [text width=0.35\textwidth, fill=Khaki, 
below=0.5cm of abrir.south west ] 
(escribir) 
{Escribir en forma secuencial 
\\
\tiny \texttt{fprintf(archivo, linea de texto, formato)}};
\node [ fill=BrickRed, minimum width=0.5\textwidth, 
below=2.5cm of abrir,  text=white]
(cerrar) {CERRAR};
\draw [->] (abrir.south) -- (escribir.north); 
\draw [->] (abrir.south) -- (leer.north); 
\draw [->] (escribir.south) -- (cerrar.north); 
\draw [->] (leer.south) -- (cerrar.north); 
\end{tikzpicture}

\end{frame}


\mode<all>
