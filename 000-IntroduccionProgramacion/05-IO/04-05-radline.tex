\subsubsection{Lectura secuencial}

\mode<article>

Muchas veces es necesario leer un archivo en forma 
secuencial, por ejemplo si se quiere recuperar 
un archivo de metadatos complejo. En estos casos
el procedimiento se hace un poco más engorroso pero 
es posible familiarizarse rápidamente. El problema
es qe muchas veces es necesario usar más de un comando.

primero se usa un comando para lectura, en \texttt{Matlab}
es \texttt{fgetl}, que lee una línea del archivo y entrega un 
\emph{string}, luego es necesario usar otro comando para 
transformar ese texto en las variables numéricas (punto flotante 
etc), en \texttt{Matlab} utilizamos \texttt{strread}. 

Son útiles otros comandos como el \texttt{feof} que detectan 
si se ha llegado al final del archivo.

En la \autoref{FigReadSequential} se muestra un ejemplo (bastante sobreactuado)
de la lectura de un archivo con toda la informción en una sola columna, separando
con etiquetas las distintas variabels guardadas.

\begin{figure}

  \includeslide[width=\textwidth]{FrameReadSequential}
  \caption{
    \protect\label{FigReadSequential}
      Ejemplo de lectura secuencial de un archivo complejo implementada 
      en \texttt{Matlab}. Sobre el gráfico se observan distintas seccines
      del archivo \texttt{tablesincos.dat} con los números de lína a la
      izquierda de los cuadros.
    }
  
\end{figure}

\mode*

\begin{frame}<presentation>[label=FrameReadSequential]
  \frametitle{Lectura secuencial}

  \begin{columns}
    \column{0.4\textwidth}
    \begin{codeblock}
      \VerbatimInput[fontsize=\tiny]{CODEXAMPLES/readcomplex.m}
    \end{codeblock}
    \column{0.6\textwidth}
    \scriptsize\texttt{tablesincos.dat}
    \begin{columns}
      \column{0.2\textwidth}
      \VerbatimInput[fontsize=\tiny, lastline=5, numbers=left,frame=single]{CODEXAMPLES/tablesincos.dat}
      \column{0.2\textwidth}
      \VerbatimInput[fontsize=\tiny, firstline=101, lastline=105, numbers=left, frame=single]{CODEXAMPLES/tablesincos.dat}
      \column{0.2\textwidth}
      \VerbatimInput[fontsize=\tiny, firstline=203, lastline=207, numbers=left, frame=single]{CODEXAMPLES/tablesincos.dat}
    \end{columns}
    \center
    \includegraphics[width=0.8\textwidth]{CODEXAMPLES/sincos.png}
  \end{columns}

\end{frame}

\mode<all>
