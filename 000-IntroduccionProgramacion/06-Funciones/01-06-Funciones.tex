\mode<article>

Normalmente se desarrolla una pieza de código con el objetivo
de poder repetir la implementación en la mayor cantidad de casos
posibles. Una \texttt{función} permite justamente reutilizar
una pieza de código generalizada en relación a datos (variables) de
entrada. La \texttt{función} entrega variables de salida que permiten
\emph{comunicar} el resultado de la operación al flujo principal
del programa. Estos conceptos se muestran en la \ref{FigFrameFunciones}

\begin{figure}

  \includeslide[width=\textwidth]{FrameFunciones}
  \caption{ Las funciones deben recibir variables de entrada
  y entregan variables de salida \protect\label{FigFrameFunciones} }

\end{figure}

\mode*
\begin{frame}<presentation>[label=FrameFunciones]
  \frametitle{Funciones}
    \includegraphics{./TIKZPICTURES/Fig-01-06-Funcion.pdf}

\end{frame}
\mode<all>
