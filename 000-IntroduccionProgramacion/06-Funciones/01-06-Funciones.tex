\mode<article>

Normalmente se desarrolla una pieza de código con el objetivo
de poder repetir la implementación en la mayor cantidad de casos
posibles. Una \texttt{función} permite justamente reutilizar
una pieza de código generalizada en relación a datos (variables) de
entrada. La \texttt{función} entrega variables de salida que permiten
\emph{comunicar} el resultado de la operación al flujo principal
del programa. Estos conceptos se muestran en la \autoref{FigFrameFunciones}

\subsection{ Funciones en Python }

En python la sintaxis es muy similar. 
La palabra clave que comienza la definición de las funciones es \texttt{def}, 
seguido del nombre de la función y los argumentos entre paréntesis. 
Los dos puntos \texttt{:} señalan el comienzo del \texttt{scope} de la función.
La palabra clave  \texttt{return } se utiliza para indicar las variables internas 
que serán devueltas a la salida de la función. 
Estos conceptos quedan ilustrados en la \autoref{FigFrameFuncionesPython}

\begin{figure}

  \includeslide[width=\textwidth]{FrameFunciones}
  \caption{ Las funciones deben recibir variables de entrada
  y entregan variables de salida \protect\label{FigFrameFunciones} }

\end{figure}

\begin{figure}

  \includeslide[width=\textwidth]{FrameFuncionesPython}
  \caption{ Las funciones deben recibir variables de entrada
  y entregan variables de salida \protect\label{FigFrameFuncionesPython} }

\end{figure}


\mode*
\begin{frame}<presentation>[label=FrameFunciones]
  \frametitle{Funciones}
    \includegraphics{./TIKZPICTURES/Fig-01-06-Funcion.pdf}

    \begin{columns}
      \column{0.6\textwidth}
	\lstinputlisting{CODEXAMPLES/funcion_partida.m}
      \column{0.4\textwidth}
	\lstinputlisting{CODEXAMPLES/use_partida.m}
    \end{columns}

\end{frame}

\begin{frame}<presentation>[label=FrameFuncionesPython]
  \frametitle{Funciones en Python}

    \begin{columns}
      \column{0.6\textwidth}
	\lstinputlisting{CODEXAMPLES/funcion_partida.py}
      \column{0.4\textwidth}
	\lstinputlisting{CODEXAMPLES/use_partida.py}
    \end{columns}


\end{frame}
\mode<all>
