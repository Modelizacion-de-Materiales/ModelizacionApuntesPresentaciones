\mode<all>
% Incluyo los paquetes necesarios.
% para no tener problemas con acentos etc.
\usepackage[utf8]{inputenc}
% en español
\usepackage[spanish]{babel}
%matemática
\usepackage{amsmath}
% este no se si hace falta pero por las dudas
\usepackage{graphicx}
% para incluir peliculas
\usepackage{multimedia}
% para usar segunda pantalla
\usepackage{pgfpages}
\usepackage{pgf}
% para hacer dibujitos
\usepackage{tikz}
\usetikzlibrary[automata,calc,arrows,decorations.pathmorphing,backgrounds,shapes,
patterns,positioning,fit,petri,overlay-beamer-styles]
\tikzstyle{every picture}+=[remember picture]
%recuadros sencillos
%\usepackage{tcolorbox}
% enumeradores intercambiables
\usepackage{enumerate}
% para subtitulos en figuras
\usepackage{subcaption}
%listings for code input
\usepackage{listings}
%verbatim input from file!
\usepackage{verbatim}
\usepackage{fancyvrb}
% para modificar los encabezados y pies de página.
%\usepackage{fancyhdr}
%\pagestyle{fancy}
\usepackage{standalone}

\definecolor{codegreen}{rgb}{0,0.6,0}
\definecolor{codegray}{rgb}{0.5,0.5,0.5}
\definecolor{codepurple}{rgb}{0.58,0,0.82}
\definecolor{backcolour}{rgb}{0.95,0.95,0.92}

\lstdefinestyle{codeblock}{
    backgroundcolor=\color{backcolour},   
    commentstyle=\color{codegreen},
    keywordstyle=\color{magenta},
    numberstyle=\tiny\color{codegray},
    stringstyle=\color{codepurple},
    basicstyle=\ttfamily\footnotesize,
    breakatwhitespace=false,         
    breaklines=true,                 
    captionpos=b,                    
    keepspaces=true,                 
    numbers=left,                    
    numbersep=5pt,                  
    showspaces=false,                
    showstringspaces=false,
    showtabs=false,                  
    tabsize=2
}


%\mode<presentation>{
% para ver las notas con la presentacion.  
%\setbeameroption{hide notes}
%para dejar las notas en la segunda patnalla
%}

% incluyo los beamercolors
\include{./PREAMBLE/BEAMERCOLORS}

%incluyo el tema y modificaciones
\include{./PREAMBLE/BEAMERTHEME}

% modifico los temas
\include{./PREAMBLE/THEMES}
% defino el template para la diapositiva del título

%%%%%%%%%%%%%%%%%%%%%%%%%%%%%%%%
% Defino la Clase
%%%%%%%%%%%%%%%%%%%%%%%%%%%%%%%%
\subtitle[Modelización 2020]{ Modelización de Propiedades y Procesos 2020 }
\author{
  Ruben Weht\inst{1,2} 
  \and Silvio Terlisky\inst{1,3}
  \and Pablo Gargano\inst{1,3} 
  \and Mariano Forti\inst{1,3} 
}
\institute{
  \inst{1}Instituto de Tecnología Prof. Jorge Sabato
  \and
  \inst{2}Fisica del Sólido, Edificio TANDAR,
  \and
  \inst{3} Gerencia Materiales
}

\mode<presentation>{\date{\scriptsize\href{mailto:model.sabato@gmail.com}{model.sabato@gmail.com}}}

\mode<article>{
  \date{
    \small
  \textsuperscript{1}Instituto de Tecnología Prof. Jorge Sabato\\
  \textsuperscript{2}Fisica del Sólido\\
  \textsuperscript{3}Gerencia Materiales\\
  \href{mailto:model.sabato@gmail.com}{model.sabato@gmail.com}
}

%defino los encabezados y pies de págna para
% dodo el documento en función de la materia y la
% clase.
%\fancyhead[L]{\tiny Modelización de Materiales 2019}
%\fancyhead[R]{\tiny \leftmark}
}

\title{
  \mode<article>{
\includegraphics[height=1cm]{./PREAMBLE/logo-isabt25.png}
\hfill
\includegraphics[height=1cm]{./PREAMBLE/ISOLOGOCNEA.png}
\hfill
\includegraphics[height=1cm]{./PREAMBLE/logo-unsam.png}
\\}  Herramientas y Elementos de Programación}
\subject{Introducción a la Materia. Repaso de elementos de Programación.}
\keywords{Matlab, Modelizacion 2019, Programación}
% Inicia el documento.
\begin{document}

% Título de la clase. 
\mode<presentation>{
\begin{frame}[plain]
\titlepage
\end{frame}
}

\mode<article>{
\maketitle
}


\mode<all>

\section{Problema Ingenieril o Físico}
\mode<article>
Cuando un equipo profesional se disponoe a resolver 
un problema Ingenieril o físico, es posible dividir la 
tarea en tres etapas, como se puestra en la Figura 
\ref{FiguraProbIngFis}.  
El \emph{preproceso} 
abarca toda la etapa de planteo del problema
desde la identificación del modelo físico a 
usar, las condiciones de contorno, toma de 
medidas, modelo matemático, etc. 

Como \emph{proceso} del problema, podemos ubicar
la etapa de implementación del modelo
matemático. Cuando se usen computadoras
para resolver esta parte del problema,
la discreitzación y matricialización de 
las funciones de estado y de las variables
independientes pueden ubicarse en esta 
etapa. La resolución, i.e. la ejecucion
del comando final , y el registro de los
resultados pueden ubicarse aquí.

Por último, en la etapa de 
\emph{postproceso} podemos ubicar todas
las tareas de medición y análisis sobre 
los resultados obtenidos en la etapa
anterior. 

 \begin{figure}
  \includeslide[width=\textwidth]{FrameProbIngFis}
  \caption{División del problema a resolver en etapas de análisis \label{FiguraProbIngFis}}

\end{figure}


\mode*

\begin{frame}<presentation>[label=FrameProbIngFis]{Problema Ingenieril o Físico}
\begin{columns}[t]
\column{0.3\textwidth}
\begin{beamercolorbox}[ht=1cm,sep=10pt]{header1}
\centering Preproceso \par
\end{beamercolorbox}
\begin{itemize}
\item “dibujo” del problema
\item recinto de validez

\item Modelo Físico\tikz\node[overlay](n1){};
\item Condiciones de contorno\tikz\node[overlay](n2){};
\end{itemize}
\column{0.3\textwidth}
\begin{beamercolorbox}[ht=1cm,sep=10pt]{header2}
\centering Proceso \par
\end{beamercolorbox}
\begin{itemize}
\item  \tikz\node[overlay](n3){}; Matricialización
\item Lectura de datos
\item Resolución
\item Escritura de resultados
\end{itemize}
\pause
\tikz[overlay] \draw[->,draw=red] (n1) -- (n3);
\tikz[overlay] \draw[->,draw=red] (n2) -- (n3);
\pause
\column{0.3\textwidth}
\begin{beamercolorbox}[ht=1cm,sep=10pt]{header3}
\centering Postproceso \par
\end{beamercolorbox}
\begin{itemize}
\item Mediciones Ingenieriles 
\item Información Gráfica
\item Interpretación de resultados
\end{itemize}
\end{columns}

\end{frame}

\mode<all>


\newcommand{\timebar}[1]{
\begin{tikzpicture}
\fill [gray] (0,0) rectangle (1.5,1); 
\fill [Cerulean] (1.5,0) rectangle (4,1); 
\draw (0,0) rectangle (4,1);
\node at (2,0.5) {\textcolor{white}{#1}};
\end{tikzpicture}
}

\mode<article>

En todas las etapas de la resolución 
del problema, el ingeniero o profesional 
hará uso de ciertas herramientas que 
le permitirán resolver el problema. 
La participación participación 
(i.e. tiempo de trabajo) estará repartida
entre el usuario y la herramientas
en sí (típicamente la pc 
o software a utilizar). 

\mode*

\begin{frame}<presentation>[fragile,label=FrameResulucionProblemas]{Resolucion de Problemas}

%\begin{columns}[t]
%\column{0.3\textwidth}
\timebar{\bfseries{  PREPROCESO }}
%\column{0.3\textwidth}
\timebar{\bfseries{  PROCESO }}
%\column{0.3\textwidth}
\timebar{\bfseries{  POSPROCESO }}
%\end{columns}

\vspace{2cm} 

\centering
\tikz\fill [Cerulean] (0,0) rectangle (0.5,0.5) ;
Usuario, Ingeniero o Profesional

\tikz\fill [gray] (0,0) rectangle (0.5,0.5);
Herramienta de Calculo

\end{frame}

\mode<all>


\section{Herramientas}
\newcommand{\userbar}[2]{
  %\userbar{text}{nodename}
\begin{tikzpicture}
  \fill [gray] (0,0) rectangle (1,1.5); 
  \node (#2) [fit={(0,0) (3,1)},text width=\textwidth,align=left] at (1.5,0.5) {\textbf{#1}};
\end{tikzpicture}
}

\newcommand{\pcbar}[4]{
  % este comando se usa: \pcbar{start}{end}{text}{nodename}
\begin{tikzpicture}
  \fill [Cerulean] (#1,0) rectangle (#2,1.5); 
  \node (#4) [fit={(0,0) (3,1)},text width=\textwidth,align=left] at (1.5,0.5) {\textbf{#3}};
\end{tikzpicture}
}

\mode<article>

En numerosas ocasiones, el profesional se encontrará con el 
hecho de que no es suficiente el uso de una única herramienta 
la solución integral de un problema. Esto se debe a que 
algunas herramientas se desarrollan con cierto grado de 
especificidad para alguna tarea. Por ejemplo, puede 
ser nercesaria una cinta métrica para medir un recinto 
y un software de elementos finitos para resolver 
el modelo computacional en el recinto medido. 

Dentro de las herramientas computacionales, es posible
usar distintos paquetes o programas en distintas etapas.
Típicamente esto se aplica a los programas para cálculo
numérico necesarios durante el proceso, y los programas
de visualización de campos escalares o vectoriales en
la etapa de postproceso. Resulta evidente entonces
que estos programas deberán asegurar la compatibilidad
en el registro de resultados para poder integrar 
el análisis. Por ejemplo, durante la materia necesitaremos
escribir los resultados de nuestros modelos de elementos
finitos en formato de texto plano con alguna sintaxis
especial para poder visualizar desplazamientos y 
fuerzas en un programa de visualización de 
mallas. 

\mode*

\begin{frame}<presentation>[label=FrameUsoHerramientas]{Uso de Herramientas}

\begin{columns}[t]
  \column{0.2\textwidth}   % columna 1
    \begin{beamercolorbox}[ht=1cm]{whitebox}
    \end{beamercolorbox}
  \begin{center}
     \includegraphics[height=1cm]{./media/mate.jpg}
     \vspace{1cm}

    \includegraphics[height=1cm]{./media/mathworks.png}
     \vspace{1.2cm}

    \includegraphics[height=1cm]{./media/GMSH.png}
  \end{center}

  \column{0.2\textwidth} % columna 2
    \begin{beamercolorbox}[ht=1cm,sep=10pt]{header1}
      \centering Preproceso \par
    \end{beamercolorbox}

    \userbar{Modelo Físico\\Condiciones de Contorno}{A1}

    \pcbar{0.7}{1.5}{Dibujo \\ discretización}{A2} %{1.5}{2.5}

    \pcbar{1.5}{3}{Visualización de \\ Resultados}{A3}
    
  \tikz\draw[overlay,->,>=latex,draw=black,thick]   (A2.south) .. controls +(-0.3,0) .. (A3.mid);
   

  \column{0.2\textwidth} % columna 3
    \begin{beamercolorbox}[ht=1cm,sep=10pt]{header2}
      \centering Proceso \par
    \end{beamercolorbox}

    \userbar{Matricializacion}{B1}

    \pcbar{1}{3}{I/O datos \\ resolución \\ I/O resultados}{B2}

    \tikz\draw[overlay,->,>=latex,draw=black,thick]    (B1) .. controls +(-0.2,-0.2) .. (B2.north);
    
  \column{0.2\textwidth} % columna 4
    \begin{beamercolorbox}[ht=1cm,sep=10pt]{header3}
      \centering Postproceso \par
    \end{beamercolorbox}

    \userbar{Interpretación de resultados}{C1}

    \pcbar{1}{2}{Mediciones}{C2}

    \pcbar{2}{3}{Gráficos}{C3}

    \tikz\draw[overlay,->,>=latex,draw=black,thick]  (C2) -- (C3);

\end{columns}
\end{frame}

\mode<all>


\mode*

\begin{frame}<presentation>[label=FrameHerramientas]{Herramientas Alternativas}

\begin{columns} 

\column{0.5\textwidth}
\begin{center}

\includegraphics[height=1cm]{./media/Matlab_Logo.png}
\LARGE Matlab \par
\vspace{1cm} 

\includegraphics[height=1cm]{./media/punch.jpg}
\LARGE Fortran \par
\vspace{1cm} 

\includegraphics[height=1cm]{./media/scipyshiny_small.png}
\LARGE Python / SciPy \par
\end{center}

\column{0.5\textwidth}

\begin{center}

\includegraphics[height=1cm]{./media/logo_octave.png}
\LARGE Octave \par
\vspace{1cm} 

\includegraphics[height=1cm]{./media/scilab.png}
\LARGE SciLab \par
\vspace{1cm} 

\onslide<2>{
  \includegraphics[height=2cm]{./media/fray.jpeg}
}
\includegraphics[height=2cm]{./media/The_C_Programming_Language_logo.png}

\end{center}

\end{columns}

\end{frame}

\mode<all>


\mode<all>
\section{Conceptos de Programación}
\newcommand\minilogo{
  \includegraphics[width=0.5cm]{./media/Matlab_Logo.png}
}

\mode<article>{
En lo que sigue daremos una revisión
de los elementos de programación 
mínimos necesarios para esta materia.
Como lenguaje oficial elegimos 
\texttt{MATLAB}, no solo por razones
históricas, sino por su accesibilidad
para profesionales sin experiencia previa
en programación. En los casos donde sea posible, 
se darán las indicaciones equivalentes 
en otros lenguajes de uso corriente en la material 


% \begin{figure}
% \includeslide{FrameVentanaMatlab}
% \caption{Ventana de Matlab. se observan 
% el arbol de archivos del directorio
% de trabajo, el editor de guiones
% y la línea de comandos. 
% \label{FigMatlabInicio} }
% \end{figure}
}

\begin{frame}<presentation>[label=FrameMatlabPrimero]{Primeros Pasos en Matlab}
\begin{itemize}
\item[\minilogo] MATrix LABoratory
\item[\minilogo] Multiplataforma
\item[\minilogo] \url{http://www.mathworks.com/products/matlab}
\item[\minilogo] Lenguaje de programación, consola progrmable, ejecucion y redacción de scripts (guiones).
\end{itemize}
\end{frame}


\begin{frame}[label=FrameVentanaMatlab]
\frametitle<presentation>{Aspecto del Escritorio de Matlab}
%\mode<presentation>{
\begin{figure}
\begin{center}
\begin{tikzpicture}
\node [anchor=south west] (image) at (0,0) {\includegraphics[width=0.7\textwidth]{./media/02-escritorio.png}};
\begin{scope}[x={(image.south east)},y={(image.north west)}]
  \node [rectangle,thick,fill=white,draw=black] at (0.15,0.2) {\small Archivos};
  \node [rectangle,thick,fill=white,draw=black] at (0.7,0.2) {\small  linea de comandos };
  \node [rectangle,thick,fill=white,draw=black] at (0.7,0.7) {\small  Editor de guiones };
\end{scope}
\end{tikzpicture}
\mode<article>{
 \caption{Ventana de Matlab. se observan 
 el arbol de archivos del directorio
 de trabajo, el editor de guiones
 y la línea de comandos. 
 \label{FigMatlabInicio}
  }
}
\end{center}
\end{figure}
%}
\mode<article>{
Busque en su pc el ícono de inicio de MATLAB, o bien inicie el programa
desde la línea de comandos. 

 Al inicar MATLAB en cualquier plataforma, se observan las ventanas de la 
 Figura \ref{FigMatlabInicio}. Revise las configuraciones para obtener
el escritorio que le sea cómodo. Sin embargo, puede encontrar de 
 utilidad un escritorio como el que se muestra. 

 Si bien especificamos las indicaciones para esta herramienta 
 en particular, la estructura general vale para cualquier 
 Entorno Integrado de Desarrollo (IDE) que pueda encontrar. 
} 
\end{frame}

%%\begin{frame}<presentation>[label=FrameInicioMatlab]{Escritorio y Consola de Matlab}
\begin{figure}
\subcaptionbox[.6\textwidth]{Ventana de inicio\label{FigMatlabVentanaInicio}}{
\includegraphics[width=0.2\textwidth]{./media/01-inicio.png}
}
\subcaptionbox[.4\textwidth]{Vista del Escritorio: árbol de directorio,
Editor, Ventana de comandos \label{FigMatlabEscritorio}}{
\includegraphics[width=0.4\textwidth]{./media/02-escritorio.png}
}
%\mode<article>{
%  \caption{Inicio del programa Matlab en cualquier plataforma\label{FigMatlabInicio}}
%}
\end{figure}

\end{frame}



\subsection{Asignación de Variables}

\mode*
\begin{frame}<presentation>[fragile,label=FrameAsignacionVariables]{Asignación de Variables .m}

\mode<trans>{
  Asignación de una matriz
}
\begin{columns}[T]
\column{0.5\textwidth}

 \vspace{0.5cm}

\hfill \texttt{Asignación de variables}

\vspace{1cm}

\hfill \texttt{Transponer},

\vspace{2cm}

  \hfill \texttt{Referencia a un elemento de la matriz (indexación).}

\vspace{0.8cm}

\hfill \texttt{prompt.}

\column{0.5\textwidth}
\begin{codeblock}
  \verbatiminput{./CODEXAMPLES/ASIGNACION.m}
\end{codeblock}

\end{columns}
\end{frame}

\begin{frame}<presentation>[label=FrameAsignacionesPython]
  \frametitle{Asignacion de Variables .py}
\mode<trans>{
  Asignación de una matriz (version Python)
}
\begin{columns}[T]
\column{0.3\textwidth}

 \vspace{0.5cm}

\flushright \texttt{Asignación de variables}

\vspace{1cm}

\hfill \texttt{Transponer},

\vspace{2cm}

  \hfill \texttt{Referencia a un elemento de la matriz (indexación).}

\vspace{0.8cm}

\hfill \texttt{prompt.}

\column{0.7\textwidth}
\begin{codeblock}
  \verbatiminput{./CODEXAMPLES/ASIGNACION.py}
\end{codeblock}

\end{columns}

\end{frame}

\mode<article>
  En cualquier lenguaje de programación, la asignación 
  de variables es la operación básica que debe aprender.
  A partir de las variables asignadas usted podrá generalizar sus
  programas para maximizar el número de casos de uso. 

  La asignación de variables se lleva a cabo mediante
  el operador \texttt{ = } de la siguente como se muestra
  en la \autoref{FigAsignacionVariables}. notar el uso 
  del punto y coma al final de la orden en la línea de 
  comandos. cuando no se usa, matlab muestra en pantalla
  el valor asignado a la variable.

\begin{figure}
  \includeslide[width=\textwidth]{FrameAsignacionVariables} 
\caption{Asignación y operaciones básicas en MATLAB\label{FigAsignacionVariables}}
\end{figure}

\begin{figure}
  \includeslide[width=\textwidth]{FrameAsignacionesPython} 
\caption{Asignación y operaciones básicas en Python. Observe la importación del 
        módulo \protect\emph{Numpy} para el tratamiento de Matrices. \protect\label{FigAsignacionVariablesPython}}
\end{figure}

\mode* 

\subsection{Indexación de Variables}

\begin{frame}<presentation>[fragile,label=FrameMatlabIndexacion]{Indexación de Variables}

\begin{columns}[T]
\column{0.25\textwidth}
  \vspace{0.5cm}
\flushright \texttt{Rango de filas, todas las columnas}

\column{0.25\textwidth}
  \textbf{Matlab}

\begin{codeblock}
  \verbatiminput{./CODEXAMPLES/Slice1.m}
\end{codeblock}
  \column{0.4\textwidth}
  \textbf{Python}
\begin{codeblock}
%  \verbatiminput{./CODEXAPLES/Slice1.py}
  \verbatiminput{./CODEXAMPLES/Slice1.py}
\end{codeblock}

\end{columns}
  \vspace{0.5cm}
\begin{columns}[T]
\column{0.25\textwidth}
\hfill \texttt{Vector de Índices}

\column{0.25\textwidth}
\begin{codeblock}
  \verbatiminput{./CODEXAMPLES/Slices2.m}
\end{codeblock}

  \column{0.4\textwidth}
  \begin{codeblock}
    \verbatiminput{./CODEXAMPLES/Slices2.py}
  \end{codeblock}
\end{columns}
\end{frame}

\mode<article>
Una vez asignada la matriz, es posible indexar sus componentes. 
Pueden referirse individualmente el elemento de la fila \texttt{i} y la
columna \texttt{j} pidiendo el elemento \texttt{A(i,j)}. Sin 
embargo, es posible realizar operaciones más complejas. Por ejemplo,
puede refirerse a un \emph{slice} de la matriz indicando un rango 
de índices en un vector, como se muestra en la
\autoref{FigMatlabIndexacion}. La idea de los slices de arrays de una o 
más dimensiones persiste en otros lenguajes, y será particularmente
útil más adelante en esta materia, por lo que se sugiere que
verifique su implementación en el lenguaje de programación 
que elija. 

Observe en el ejemplo de la \autoref{FigMatlabIndexacion} para \emph{Python}
la necesidad de usar la función \texttt{ Numpy.ix\_ } para generar todas las 
combinaciones de índices dadas por los índices de las columnas 
\texttt{ ( i\textsubscript{1} , i \textsubscript{2} ) }
y de las columnas \texttt{ ( j\textsubscript{1} , j \textsubscript{2} ) }, 
mientras que en matlab no es necesario
una función extra. Verifique en su lenguaje de programación cómo debe indexar 
las filas y las columnas para obtener el resultado del ejeplo. 

\mode*

\begin{figure}
  \includeslide[width=\textwidth]{FrameMatlabIndexacion}
  \caption{ Indexación de Matrices con utilización de lístas de índices para
 obtener un slice de la matriz \label{FigMatlabIndexacion} }
\end{figure}

\mode*

\subsection{Sentencias de Control de Flujo: Lazos}

\begin{frame}<presentation>[label=FrameMatlabForWhile]{Control de Flujo}
  \begin{columns}
    \column{0.2\textwidth}
    \hfill

    \column{0.4\textwidth}
    \center{ \textbf{Matlab} }

    \column{0.4\textwidth}
    \center{\textbf{Python}}
  \end{columns}

  \begin{columns}[T]
    \column{0.2\textwidth}
      \flushright \large\texttt{for}

    \column{0.4\textwidth}
      \begin{codeblock}
	\verbatiminput{./CODEXAMPLES/suma_for.m}
      \end{codeblock}

    \column{0.4\textwidth}
      \begin{codeblock}
	\verbatiminput{./CODEXAMPLES/suma_for.py}

      \end{codeblock}

  \end{columns}

  \begin{columns}[T]
    \column{0.2\textwidth}
    \flushright \large\texttt{while}

    \column{0.4\textwidth}
      \begin{codeblock}
	\verbatiminput{./CODEXAMPLES/suma_while.m}
      \end{codeblock}

    \column{0.4\textwidth}
      \begin{codeblock}
	\verbatiminput{./CODEXAMPLES/suma_while.py}
      \end{codeblock}

  \end{columns}
\end{frame}

\mode<article>

Frecuentemente se encontrará con la necesidad de repetir una serie de 
comandos un número fintio de veces o bien hasta que se cumpla alguna 
condición lógica. Las Sentencias de Control de flujo que se usan en esas 
ocaciones son el \textbf{for} y el \textbf{while}, respectivamente, 
cuyas sintaxis se muestran en la \autoref{FigMatlabForWhile}. A la pieza de
código que forman estas sentencias se las conoce como
lazos o bucles (loop).

\begin{figure}
  \includeslide[width=\textwidth]{FrameMatlabForWhile}
  \caption{
    Sintaxis para las sentencias for y while
  para el control de flujo de ejecución \label{FigMatlabForWhile}
}
\end{figure}

En el caso de usar la sentencia \textbf{for}, los valores que debe tomar
el iterador se dan en un rango de la forma  $[ i_{min} :  \Delta i : i_{max} ]$, 
estos valores enteros positivos. Nuevamente se ha dado la sintaxis en matlab
pero pueden encontrarse formas equivalentes en otros lenguajes. 

Luego, en el caso de utilizar la sentencia \textbf{while}, el criterio 
de ejecución se da con una sentencia lógica, por ejemplo la comparación
entre dos números reales, o simplemente la evaluación de una 
variable tipo lógico. Si se necesita un contador, en estos casos
debe actualizarse el valor del mismo en forma manual. Es buena
práctica poner además una condicón que fuerce la terminación  
del lazo en caso de que la condición lógica no se cumpla 
en una cantidad de iteraciones razonable. 

\mode*

\subsection{Control de Flujo: Condicionales}

\begin{frame}<presentation>[label=FrameMatlabIf]{Control de flujo: Condicionales} 
\begin{columns}[T]
\column{0.2\textwidth}
\hfill \large\texttt{if}

\column{0.4\textwidth}
\begin{codeblock}
\verbatiminput{./CODEXAMPLES/suma.m}
\end{codeblock}

\column{0.4\textwidth}
\begin{codeblock}
\verbatiminput{./CODEXAMPLES/suma_continue.m}
\end{codeblock}

\end{columns}
\end{frame}

\mode<article>

Otro caso de trabajo frecuente es el de tener 
la necesidad de condicionar la ejecución de 
una pieza de código a la ocurrencia de una 
condición lógica. Para esto se usa la sentencia
\textbf{if}. La estructura más general \textbf{if},
\textbf{else if}, \textbf{else} , \textbf{end} evalúa
una serie de condiciones en forma posicional  y 
excluyente, como se esquematiza en la 
\autoref{FigMatlabIf}. Las sentencias \textbf{else if} y 
\textbf{else} son opcionales. Si la condición 
de argunento en \textbf{if} se cumple, se ejecuta
la serie de comandos hasta \textbf{else} o 
\textbf{end}, lo que se encuentre primero. Al 
terminarse la serie de comandos, no se evalúan 
las condiciones que son argumento de los 
siguientes \textbf{else if}, sino que se
prosigue a partir de end. 

\begin{figure}
  \includeslide[width=\textwidth]{FrameMatlabIf}
  
  \caption{Esquema de la aplicación de los condicionales   para 
  evaluar piezas de código sujetas a condiciones 
  particulares o generales. Notar el uso de los comandos 
  \protect\textbf{continue} y \protect\textbf{break} para forzar la 
  omisión en un lazo o la salida del mismo
  \label{FigMatlabIf}
  }

\end{figure}

Si la condición lógica que es argumento de \textbf{if}
no se cumple, se evalúa 
la condición lógica que es argunmento del 
primer \textbf{else if}, si ésta no se cumple 
se siguen evaluendo los argumentos de las
sentencias \textbf{else if} hasta que se encuentra 
una condición verdadera. En ese caso ocurre lo 
mismo que se explicó antes: cuando se termina
de ejecutar la serie de comandos habilitados
por esta instancia de \textbf{else if}, se
prosigue a partir de \textbf{end}.

La sentencia \textbf{else} da lugar a una serie de
comandos que se ejecutarán solo si ninguna de las 
condiciones lógicas evaluadas es verdadera. 
Podría decirse que marca lo que debe ejecutarse por 
descarte de las condiciones evaluadas. 

En forma general, puede tenerse en cuenta que las 
condiciones sobre las sentencias \textbf{if}, 
\textbf{else if} \emph{ detectan} casos especiales
mientras que \textbf{else} da lugar a la ejecución
sobre el caso más general pero en forma 
excluyente de los casos particulares. Este 
concepto nos será de utilidad en la separación
de las condiciones de contorno sobre un 
recinto de integración para la solución
de ecuaciones diferenciales. 

\mode*

\mode<all>
\subsection{Entrada y Salida de datos}

\mode<article>

El registro de los resultados de un problema es tan 
importante como la trazabilidad de los datos 
que se tuvieron en cuenta para resolverlo. Por lo tanto, 
en general se guardan ambos tipos de datos en archivos, 
que serán escritos y leídos por nuestra herramienta 
informática oportunamente. Notar entonces que la legibilidad
y la compatibilidad con otras herramientas 
serán características obligadas de los archivos
de entrada y salida generados.

La secuencia de comandos que deben usarse para estos
fines responden al diagrama de flujo de la 
\autoref{FigFlujoIO}. Siempre habrá un comando para 
abrir el archivo, habrá comandos para leer (o 
escribir) en forma secuencial  el archivo de entrada
(o de salida), y por último se ejecutará un comando 
para cerrar el archivo.

\begin{figure}
  \includeslide[width=\textwidth]{FrameEstructuraIO}
\caption{Flujo de ejecución para la secuencia de 
lectura/escritura de los archivos de registro de resultados
y de datos de entrada \label{FigFlujoIO}.}
\end{figure}

\mode*


\begin{frame}<presentation>[label=FrameEstructuraIO]
\frametitle{Input-Output / Entrada-Salida de Datos}
\tikzstyle{every node}=[draw, line width=2pt,
  outer sep=0.1cm, align=center,text height=8pt]
\tikzstyle{every path}=[line width=2pt, >=latex]
\begin{tikzpicture}
\node [ minimum width=\textwidth, fill=Purple, draw, 
line width=2pt, text=white] (title)  { Estructura IO};
\node [ minimum width=0.5\textwidth, fill=BrickRed, 
below=1cm of title, text=white ]
(abrir) {ABRIR};
\node [ text width=0.35\textwidth, fill=Khaki, 
below=0.5cm of abrir.south east ] (leer) 
{Leer en forma secuencial 
\\
\tiny \texttt{linea = read(archivo)}
};
\node [text width=0.35\textwidth, fill=Khaki, 
below=0.5cm of abrir.south west ] 
(escribir) 
{Escribir en forma secuencial 
\\
\tiny \texttt{fprintf(archivo, linea de texto, formato)}};
\node [ fill=BrickRed, minimum width=0.5\textwidth, 
below=2.5cm of abrir,  text=white]
(cerrar) {CERRAR};
\draw [->] (abrir.south) -- (escribir.north); 
\draw [->] (abrir.south) -- (leer.north); 
\draw [->] (escribir.south) -- (cerrar.north); 
\draw [->] (leer.south) -- (cerrar.north); 
\end{tikzpicture}

\end{frame}


\mode<all>

\subsubsection{Apertura y cierre de archivos}
\mode<article>

En todos los lenguajes de programación se utiliza algún comando
para abrir los archivos, por ejemplo \textbf{fopen} en matlab.
Este comando suele aceptar un argumento para indicar el
nombre de archivo a abrir, que puede ser una variable 
de tipo string o cadena de caracteres. Otro argumento
que se le da a \textbf{fopen} es el permiso con el 
que el archivo debe ser abierto, por ejemplo
de escritura (generalmente \textbf{w}, solo lectura \textbf{r}
, o para agregar contenido al final o \emph{append} \textbf{a}. 
En la  \autoref{FigIOAbrirArchivos} se esquematiza
la utilización de estos comandos. 

Es necesario notar que el comando \textbf{open} devuelve
una variable de salida que en el ejemplo se asigna 
en la variable \textbf{fid}. 
La importancia de la misma radica en que a partir
de la asignación \textbf{fid} se convierte en 
un \emph{indicador del archivo}, una especie de puntero
a la primer línea disponible del archivo. En caso
de haberlo abierto con permisos de escritura, 
el puntero indica la primer línea del 
archivo que debe escribirse. Por el contrario
al abrir el archivo con permisos de lectura
el puntero apunta a la primer
línea que puede leerse. El concepto quedará 
claro enseguida. 

\begin{figure}
\includeslide[width=\textwidth]{FrameIOAbrirArchivos}
\caption{Comandos básicos para la apertura de un archivo\label{FigIOAbrirArchivos}}
\end{figure}

\mode*

\begin{frame}<presentation>[label=FrameIOAbrirArchivos]
\frametitle{Apertura y cierre de archivos}

\begin{columns}[T]
\column{0.5\textwidth}
\hfill Abrir archivo existente para lectura:
\column{0.5\textwidth}
\begin{codeblock}
\verbatiminput{./CODEXAMPLES/open_file.m}
\end{codeblock}
\end{columns}

\begin{columns}[T]
\column{0.5\textwidth}
\hfill Abrir archivo \emph{inexistente} para lectura:
\column{0.5\textwidth}
\begin{codeblock}
\verbatiminput{./CODEXAMPLES/open_file_noex.m}
\end{codeblock}
\end{columns}

\begin{columns}[T]
\column{0.5\textwidth}
\hfill Abrir otro archivo  para \emph{escritura}:
\column{0.5\textwidth}
\begin{codeblock}
\verbatiminput{./CODEXAMPLES/open_otherfile.m}
\end{codeblock}
\end{columns}

\end{frame}

\mode<all>

\subsection{Escritura Secuencial}
\mode<article>

Una estrategia para registrar los datos en un archivo es 
escribir en el mismo línea por línea, o en forma \emph{secuencial}.
Una vez abierto el archivo con permisos para escritura, 
se escriben los datos con el comando \textbf{fprintf}. 
El mismo acepta como argumentos el indicador de archivo, 
un \emph{string} de formato y la lista de variables 
que se guardan. El mecanismo se ilustra en la 
\autoref{FigIOEscribir}.

El string de formato suele usar caracteres especiales para indicar
la posición en la que la lista de variables debe 
imprimirse. No daremos detalle aquí, pero puede
buscar en los manuales el uso de los \emph{format 
identifiers, indicadores de formato}, ya que serán 
de suma utilidad. la correspondencia entre el 
indicador de formato y la variable que se escribe
es posicional, y en general para escribir un número 
real se usa el indicador \textbf{\%f} con algún 
indicador extra para la precisión. 
Observe la necesidad de imprimir los cambios de línea
o \emph{retornos de carro} para indicar los fines de línea. 

Los archivos siempre deben ser cerrados al terminar la operación 
de escritura o lectura. 

\begin{figure}
  \includeslide[width=\textwidth]{FrameIOEscribir}
  \caption{Algunos ejemplos de escritura secuencial con formato\label{FigIOEscribir}}
\end{figure}
\mode*

\begin{frame}<presentation>[label=FrameIOEscribir]
\frametitle{Escritura línea por línea}

\begin{columns}[T]
\column{0.6\textwidth}
\hfill \small \texttt{Escribir con formato de punto flotante.} 
\tikz\node  (A) at (-0.3,0) {};
\par

  \vspace{0.5cm}
\hfill \small \texttt{Escribir con retorno de carro.}
\tikz\node  (B) at (-0.3,0) {};
\par

  \vspace{0.5cm}
\hfill \small \texttt{Escribir con formato de punto flotante.}
\tikz\node  (C) at (-0.3,0) {};
\par

  \column{0.4\textwidth}
  \tikz[overlay]\node   (X) at (-0.1,-1.5) {};
  \tikz[overlay]\node   (Y) at (-0.1,-2.0) {};
  \tikz[overlay]\node   (Z) at (-0.1,-2.8) {};
\begin{codeblock}
  \verbatiminput{./CODEXAMPLES/frprintfs.m}
\end{codeblock}

\end{columns}


\begin{columns}[T]
\column{0.5\textwidth}
 \hfill \small Resultado\par
\column{0.5\textwidth}
\begin{codeblock}
\verbatiminput{./CODEXAMPLES/datasimple.dat}
\end{codeblock}
\end{columns}

\begin{tikzpicture}[overlay]
  \draw[->,>=latex,draw,thick] (A) -- (X) ;
  \draw[->,>=latex,draw,thick] (B) -- (Y) ;
  \draw[->,>=latex,draw,thick] (C) -- (Z) ;
\end{tikzpicture}

\end{frame}

\begin{frame}<presentation>[label=FrameIOEscribirPython]
\frametitle{Apertura y cierre de archivos en python}

\begin{columns}[T]
\column{0.3\textwidth}
    \hfill Definir \texttt{strings} con formato de punto flotante.
\column{0.7\textwidth}
    \lstinputlisting[language=Python]{./CODEXAMPLES/writelines.py}
\end{columns}

    \begin{columns}[T]
        \column{0.3\textwidth}
        \hfill \texttt{data.dat}
        \column{0.7\textwidth}
        \lstinputlisting{./CODEXAMPLES/data2.dat}
    \end{columns}

\end{frame}

\mode<all>

\subsection{Funciones}
\mode<article>

Normalmente se desarrolla una pieza de código con el objetivo
de poder repetir la implementación en la mayor cantidad de casos
posibles. Una \texttt{función} permite justamente reutilizar
una pieza de código generalizada en relación a datos (variables) de
entrada. La \texttt{función} entrega variables de salida que permiten
\emph{comunicar} el resultado de la operación al flujo principal
del programa. Estos conceptos se muestran en la \autoref{FigFrameFunciones}

\subsection{ Funciones en Python }

En python la sintaxis es muy similar. 
La palabra clave que comienza la definición de las funciones es \texttt{def}, 
seguido del nombre de la función y los argumentos entre paréntesis. 
Los dos puntos \texttt{:} señalan el comienzo del \texttt{scope} de la función.
La palabra clave  \texttt{return } se utiliza para indicar las variables internas 
que serán devueltas a la salida de la función. 
Estos conceptos quedan ilustrados en la \autoref{FigFrameFuncionesPython}

\begin{figure}

  \includeslide[width=\textwidth]{FrameFunciones}
  \caption{ Las funciones deben recibir variables de entrada
  y entregan variables de salida \protect\label{FigFrameFunciones} }

\end{figure}

\begin{figure}

  \includeslide[width=\textwidth]{FrameFuncionesPython}
  \caption{ Las funciones deben recibir variables de entrada
  y entregan variables de salida \protect\label{FigFrameFuncionesPython} }

\end{figure}


\mode*
\begin{frame}<presentation>[label=FrameFunciones]
  \frametitle{Funciones}
    \includegraphics{./TIKZPICTURES/Fig-01-06-Funcion.pdf}

    \begin{columns}
      \column{0.6\textwidth}
	\lstinputlisting{CODEXAMPLES/funcion_partida.m}
      \column{0.4\textwidth}
	\lstinputlisting{CODEXAMPLES/use_partida.m}
    \end{columns}

\end{frame}

\begin{frame}<presentation>[label=FrameFuncionesPython]
  \frametitle{Funciones en Python}

    \begin{columns}
      \column{0.6\textwidth}
	\lstinputlisting{CODEXAMPLES/funcion_partida.py}
      \column{0.4\textwidth}
	\lstinputlisting{CODEXAMPLES/use_partida.py}
    \end{columns}


\end{frame}
\mode<all>


\end{document}
