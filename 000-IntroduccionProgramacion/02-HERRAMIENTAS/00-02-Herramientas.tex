\newcommand{\userbar}[2]{
  %\userbar{text}{nodename}
\begin{tikzpicture}
  \fill [gray] (0,0) rectangle (1,1.5); 
  \node (#2) [fit={(0,0) (3,1)},text width=\textwidth,align=left] at (1.5,0.5) {\textbf{#1}};
\end{tikzpicture}
}

\newcommand{\pcbar}[4]{
  % este comando se usa: \pcbar{start}{end}{text}{nodename}
\begin{tikzpicture}
  \fill [Cerulean] (#1,0) rectangle (#2,1.5); 
  \node (#4) [fit={(0,0) (3,1)},text width=\textwidth,align=left] at (1.5,0.5) {\textbf{#3}};
\end{tikzpicture}
}

\mode<article>

En numerosas ocasiones, el profesional se encontrará con el 
hecho de que no es suficiente el uso de una única herramienta 
la solución integral de un problema. Esto se debe a que 
algunas herramientas se desarrollan con cierto grado de 
especificidad para alguna tarea. Por ejemplo, puede 
ser nercesaria una cinta métrica para medir un recinto 
y un software de elementos finitos para resolver 
el modelo computacional en el recinto medido. 

Dentro de las herramientas computacionales, es posible
usar distintos paquetes o programas en distintas etapas.
Típicamente esto se aplica a los programas para cálculo
numérico necesarios durante el proceso, y los programas
de visualización de campos escalares o vectoriales en
la etapa de postproceso. Resulta evidente entonces
que estos programas deberán asegurar la compatibilidad
en el registro de resultados para poder integrar 
el análisis. Por ejemplo, durante la materia necesitaremos
escribir los resultados de nuestros modelos de elementos
finitos en formato de texto plano con alguna sintaxis
especial para poder visualizar desplazamientos y 
fuerzas en un programa de visualización de 
mallas. 

\mode*

\begin{frame}<presentation>[label=FrameUsoHerramientas]{Uso de Herramientas}

\begin{columns}[t]
  \column{0.2\textwidth}   % columna 1
    \begin{beamercolorbox}[ht=1cm]{whitebox}
    \end{beamercolorbox}
  \begin{center}
     \includegraphics[height=1cm]{./media/mate.jpg}
     \vspace{1cm}

    \includegraphics[height=1cm]{./media/mathworks.png}
     \vspace{1.2cm}

    \includegraphics[height=1cm]{./media/GMSH.png}
  \end{center}

  \column{0.2\textwidth} % columna 2
    \begin{beamercolorbox}[ht=1cm,sep=10pt]{header1}
      \centering Preproceso \par
    \end{beamercolorbox}

    \userbar{Modelo Físico\\Condiciones de Contorno}{A1}

    \pcbar{0.7}{1.5}{Dibujo \\ discretización}{A2} %{1.5}{2.5}

    \pcbar{1.5}{3}{Visualización de \\ Resultados}{A3}
    
  \tikz\draw[overlay,->,>=latex,draw=black,thick]   (A2.south) .. controls +(-0.3,0) .. (A3.mid);
   

  \column{0.2\textwidth} % columna 3
    \begin{beamercolorbox}[ht=1cm,sep=10pt]{header2}
      \centering Proceso \par
    \end{beamercolorbox}

    \userbar{Matricializacion}{B1}

    \pcbar{1}{3}{I/O datos \\ resolución \\ I/O resultados}{B2}

    \tikz\draw[overlay,->,>=latex,draw=black,thick]    (B1) .. controls +(-0.2,-0.2) .. (B2.north);
    
  \column{0.2\textwidth} % columna 4
    \begin{beamercolorbox}[ht=1cm,sep=10pt]{header3}
      \centering Postproceso \par
    \end{beamercolorbox}

    \userbar{Interpretación de resultados}{C1}

    \pcbar{1}{2}{Mediciones}{C2}

    \pcbar{2}{3}{Gráficos}{C3}

    \tikz\draw[overlay,->,>=latex,draw=black,thick]  (C2) -- (C3);

\end{columns}
\end{frame}

\mode<all>
