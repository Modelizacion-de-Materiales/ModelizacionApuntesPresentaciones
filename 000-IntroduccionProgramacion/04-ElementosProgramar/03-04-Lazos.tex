\subsection{Sentencias de Control de Flujo: Lazos}

\mode<article>

Frecuentemente se encontrará con la necesidad de repetir una serie de 
comandos un número fintio de veces o bien hasta que se cumpla alguna 
condición lógica. Las Sentencias de Control de flujo que se usan en esas 
ocaciones son el \textbf{for} y el \textbf{while}, respectivamente, 
cuyas sintaxis se muestran en la \autoref{FigMatlabForWhile}. A la pieza de
código que forman estas sentencias se las conoce como
lazos o bucles (loop).

\begin{figure}
  \includeslide[width=\textwidth]{FrameMatlabForWhile}
  \caption{
    Sintaxis para las sentencias for y while
  para el control de flujo de ejecución \label{FigMatlabForWhile}
}
\end{figure}

En el caso de usar la sentencia \textbf{for}, los valores que debe tomar
el iterador se dan en un rango de la forma  $[ i_{min} :  \Delta i : i_{max} ]$, 
estos valores enteros positivos. Nuevamente se ha dado la sintaxis en matlab
pero pueden encontrarse formas equivalentes en otros lenguajes. 

Luego, en el caso de utilizar la sentencia \textbf{while}, el criterio 
de ejecución se da con una sentencia lógica, por ejemplo la comparación
entre dos números reales, o simplemente la evaluación de una 
variable tipo lógico. Si se necesita un contador, en estos casos
debe actualizarse el valor del mismo en forma manual. Es buena
práctica poner además una condicón que fuerce la terminación  
del lazo en caso de que la condición lógica no se cumpla 
en una cantidad de iteraciones razonable. 

\mode*

\begin{frame}<presentation>[label=FrameMatlabForWhile]{Control de Flujo}
  \begin{columns}
    \column{0.2\textwidth}
    \hfill

    \column{0.4\textwidth}
    \center{ \textbf{Matlab} }

    \column{0.4\textwidth}
    \center{\textbf{Python}}
  \end{columns}

  \begin{columns}[T]
    \column{0.2\textwidth}
      \flushright \large\texttt{for}

    \column{0.4\textwidth}
      \begin{codeblock}
	\verbatiminput{./CODEXAMPLES/suma_for.m}
      \end{codeblock}

    \column{0.4\textwidth}
      \begin{codeblock}
	\verbatiminput{./CODEXAMPLES/suma_for.py}

      \end{codeblock}

  \end{columns}

  \begin{columns}[T]
    \column{0.2\textwidth}
    \flushright \large\texttt{while}

    \column{0.4\textwidth}
      \begin{codeblock}
	\verbatiminput{./CODEXAMPLES/suma_while.m}
      \end{codeblock}

    \column{0.4\textwidth}
      \begin{codeblock}
	\verbatiminput{./CODEXAMPLES/suma_while.py}
      \end{codeblock}

  \end{columns}
\end{frame}

\mode<all>

