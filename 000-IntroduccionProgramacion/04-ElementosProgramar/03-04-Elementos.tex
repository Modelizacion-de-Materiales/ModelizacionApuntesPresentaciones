
\subsection{Asignación de Variables}

\mode*
\begin{frame}<presentation>[fragile,label=FrameAsignacionVariables]{Asignación de Variables .m}

\mode<trans>{
  Asignación de una matriz
}
\begin{columns}[T]
\column{0.5\textwidth}

 \vspace{0.5cm}

\hfill \texttt{Asignación de variables}

\vspace{1cm}

\hfill \texttt{Transponer},

\vspace{1cm}

\hfill \texttt{prompt.}

\column{0.5\textwidth}
\begin{codeblock}
  \verbatiminput{./CODEXAMPLES/ASIGNACION.m}
\end{codeblock}

\end{columns}
\end{frame}

\begin{frame}<presentation>[label=FrameAsignacionesPython]
  \frametitle{Asignación de Variables .py}
\mode<trans>{
  Asignación de una matriz (versión Python)
}
\begin{columns}[T]
\column{0.3\textwidth}

 \vspace{0.5cm}

\flushright \texttt{Asignación de variables}

\vspace{1cm}

\hfill \texttt{Transponer},

\vspace{1cm}

\hfill \texttt{prompt.}

\column{0.7\textwidth}
\begin{codeblock}
  \verbatiminput{./CODEXAMPLES/ASIGNACION.py}
\end{codeblock}

\end{columns}

\end{frame}

\mode<article>
  En cualquier lenguaje de programación, la asignación 
  de variables es la operación básica que debe aprender.
  A partir de las variables asignadas usted podrá generalizar sus
  programas para maximizar el número de casos de uso. 

  La asignación de variables se lleva a cabo mediante
  el operador \texttt{ = } de la siguiente como se muestra
  en la \autoref{FigAsignacionVariables}. notar el uso 
  del punto y coma al final de la orden en la línea de 
  comandos. cuando no se usa, matlab muestra en pantalla
  el valor asignado a la variable.

\begin{figure}
  \includeslide[width=\textwidth]{FrameAsignacionVariables} 
\caption{Asignación y operaciones básicas en MATLAB\label{FigAsignacionVariables}}
\end{figure}

\begin{figure}
  \includeslide[width=\textwidth]{FrameAsignacionesPython} 
\caption{Asignación y operaciones básicas en Python. Observe la importación del 
        módulo \protect\emph{Numpy} para el tratamiento de Matrices. \protect\label{FigAsignacionVariablesPython}}
\end{figure}

\mode* 

\subsection{Indexación de Variables}

\begin{frame}<presentation>[fragile,label=FrameMatlabIndexacion]{Indexación de Variables}

\begin{columns}[T]
\column{0.25\textwidth}
  \vspace{0.5cm}
\flushright \texttt{Rango de filas, todas las columnas}

\column{0.25\textwidth}
  \textbf{Matlab}

\begin{codeblock}
  \verbatiminput{./CODEXAMPLES/Slice1.m}
\end{codeblock}
  \column{0.4\textwidth}
  \textbf{Python}
\begin{codeblock}
%  \verbatiminput{./CODEXAPLES/Slice1.py}
  \verbatiminput{./CODEXAMPLES/Slice1.py}
\end{codeblock}

\end{columns}
  \vspace{0.5cm}
\begin{columns}[T]
\column{0.25\textwidth}
\hfill \texttt{Vector de Índices}

\column{0.25\textwidth}
\begin{codeblock}
  \verbatiminput{./CODEXAMPLES/Slices2.m}
\end{codeblock}

  \column{0.4\textwidth}
  \begin{codeblock}
    \verbatiminput{./CODEXAMPLES/Slices2.py}
  \end{codeblock}
\end{columns}
\end{frame}

\mode<article>
Una vez asignada la matriz, es posible indexar sus componentes. 
Pueden referirse individualmente el elemento de la fila \texttt{i} y la
columna \texttt{j} pidiendo el elemento \texttt{A(i,j)}. Sin 
embargo, es posible realizar operaciones más complejas. Por ejemplo,
puede referirse a un \emph{slice} de la matriz indicando un rango 
de índices en un vector, como se muestra en la
\autoref{FigMatlabIndexacion}. La idea de los slices de arrays de una o 
más dimensiones persiste en otros lenguajes, y será particularmente
útil más adelante en esta materia, por lo que se sugiere que
verifique su implementación en el lenguaje de programación 
que elija. 

Observe en el ejemplo de la \autoref{FigMatlabIndexacion} para \emph{Python}
la necesidad de usar la función \texttt{ Numpy.ix\_ } para generar todas las 
combinaciones de índices dadas por los índices de las columnas 
\texttt{ ( i\textsubscript{1} , i \textsubscript{2} ) }
y de las columnas \texttt{ ( j\textsubscript{1} , j \textsubscript{2} ) }, 
mientras que en matlab no es necesario
una función extra. Verifique en su lenguaje de programación cómo debe indexar 
las filas y las columnas para obtener el resultado del ejemplo. 

\mode*

\begin{figure}
  \includeslide[width=\textwidth]{FrameMatlabIndexacion}
  \caption{ Indexación de Matrices con utilización de listas de índices para
 obtener un slice de la matriz \label{FigMatlabIndexacion} }
\end{figure}

\mode*

\subsection{Sentencias de Control de Flujo: Lazos}

\begin{frame}<presentation>[label=FrameMatlabForWhile]{Control de Flujo}
  \begin{columns}
    \column{0.2\textwidth}
    \hfill

    \column{0.4\textwidth}
    \center{ \textbf{Matlab} }

    \column{0.4\textwidth}
    \center{\textbf{Python}}
  \end{columns}

  \begin{columns}[T]
    \column{0.2\textwidth}
      \flushright \large\texttt{for}

    \column{0.4\textwidth}
      \begin{codeblock}
	\verbatiminput{./CODEXAMPLES/suma_for.m}
      \end{codeblock}

    \column{0.4\textwidth}
      \begin{codeblock}
	\verbatiminput{./CODEXAMPLES/suma_for.py}

      \end{codeblock}

  \end{columns}

  \begin{columns}[T]
    \column{0.2\textwidth}
    \flushright \large\texttt{while}

    \column{0.4\textwidth}
      \begin{codeblock}
	\verbatiminput{./CODEXAMPLES/suma_while.m}
      \end{codeblock}

    \column{0.4\textwidth}
      \begin{codeblock}
	\verbatiminput{./CODEXAMPLES/suma_while.py}
      \end{codeblock}

  \end{columns}
\end{frame}

\mode<article>

Frecuentemente se encontrará con la necesidad de repetir una serie de 
comandos un número finito de veces o bien hasta que se cumpla alguna 
condición lógica. Las Sentencias de Control de flujo que se usan en esas 
ocasiones son el \textbf{for} y el \textbf{while}, respectivamente, 
cuyas sintaxis se muestran en la \autoref{FigMatlabForWhile}. A la pieza de
código que forman estas sentencias se las conoce como
lazos o bucles (loop).

\begin{figure}
  \includeslide[width=\textwidth]{FrameMatlabForWhile}
  \caption{
    Sintaxis para las sentencias for y while
  para el control de flujo de ejecución \label{FigMatlabForWhile}
}
\end{figure}

En el caso de usar la sentencia \textbf{for}, los valores que debe tomar
el iterador se dan en un rango de la forma  $[ i_{min} :  \Delta i : i_{max} ]$, 
estos valores enteros positivos. Nuevamente se ha dado la sintaxis en matlab
pero pueden encontrarse formas equivalentes en otros lenguajes. 

Luego, en el caso de utilizar la sentencia \textbf{while}, el criterio 
de ejecución se da con una sentencia lógica, por ejemplo la comparación
entre dos números reales, o simplemente la evaluación de una 
variable tipo lógico. Si se necesita un contador, en estos casos
debe actualizarse el valor del mismo en forma manual. Es buena
práctica poner además una condición que fuerce la terminación  
del lazo en caso de que la condición lógica no se cumpla 
en una cantidad de iteraciones razonable. 

\mode*

\subsection{Control de Flujo: Condicionales}

\begin{frame}<presentation>[label=FrameMatlabIf]{Control de flujo: Condicionales} 
\begin{columns}[T]
\column{0.2\textwidth}
\hfill \large\texttt{if}

\column{0.4\textwidth}
\begin{codeblock}
\verbatiminput{./CODEXAMPLES/suma.m}
\end{codeblock}

\column{0.4\textwidth}
\begin{codeblock}
\verbatiminput{./CODEXAMPLES/suma_continue.m}
\end{codeblock}

\end{columns}
\end{frame}

\mode<article>

Otro caso de trabajo frecuente es el de tener 
la necesidad de condicionar la ejecución de 
una pieza de código a la ocurrencia de una 
condición lógica. Para esto se usa la sentencia
\textbf{if}. La estructura más general \textbf{if},
\textbf{else if}, \textbf{else} , \textbf{end} evalúa
una serie de condiciones en forma posicional  y 
excluyente, como se esquematiza en la 
\autoref{FigMatlabIf}. Las sentencias \textbf{else if} y 
\textbf{else} son opcionales. Si la condición 
de argumento en \textbf{if} se cumple, se ejecuta
la serie de comandos hasta \textbf{else} o 
\textbf{end}, lo que se encuentre primero. Al 
terminarse la serie de comandos, no se evalúan 
las condiciones que son argumento de los 
siguientes \textbf{else if}, sino que se
prosigue a partir de end. 

\begin{figure}
  \includeslide[width=\textwidth]{FrameMatlabIf}
  
  \caption{Esquema de la aplicación de los condicionales   para 
  evaluar piezas de código sujetas a condiciones 
  particulares o generales. Notar el uso de los comandos 
  \protect\textbf{continue} y \protect\textbf{break} para forzar la 
  omisión en un lazo o la salida del mismo
  \label{FigMatlabIf}
  }

\end{figure}

Si la condición lógica que es argumento de \textbf{if}
no se cumple, se evalúa 
la condición lógica que es argumento del 
primer \textbf{else if}, si ésta no se cumple 
se siguen evaluando los argumentos de las
sentencias \textbf{else if} hasta que se encuentra 
una condición verdadera. En ese caso ocurre lo 
mismo que se explicó antes: cuando se termina
de ejecutar la serie de comandos habilitados
por esta instancia de \textbf{else if}, se
prosigue a partir de \textbf{end}.

La sentencia \textbf{else} da lugar a una serie de
comandos que se ejecutarán solo si ninguna de las 
condiciones lógicas evaluadas es verdadera. 
Podría decirse que marca lo que debe ejecutarse por 
descarte de las condiciones evaluadas. 

En forma general, puede tenerse en cuenta que las 
condiciones sobre las sentencias \textbf{if}, 
\textbf{else if} \emph{ detectan} casos especiales
mientras que \textbf{else} da lugar a la ejecución
sobre el caso más general pero en forma 
excluyente de los casos particulares. Este 
concepto nos será de utilidad en la separación
de las condiciones de contorno sobre un 
recinto de integración para la solución
de ecuaciones diferenciales. 

\mode*
