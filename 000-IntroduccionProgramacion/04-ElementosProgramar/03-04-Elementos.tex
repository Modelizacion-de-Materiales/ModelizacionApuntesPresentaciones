
\subsection{Asignación de Variables}

\mode<article>
  En cualquier lenguaje de programación, la asignación 
  de variables es la operación básica que debe aprender.
  A partir de las variables asignadas usted podrá generalizar sus
  programas para maximizar el número de casos de uso. 

  La asignación de variables se lleva a cabo mediante
  el operador \texttt{ = } de la siguiente como se muestra
  en la \autoref{FigAsignacionVariables}. notar el uso 
  del punto y coma al final de la orden en la línea de 
  comandos. cuando no se usa, \texttt{Matlab} muestra en pantalla
  el valor asignado a la variable.

\begin{figure}
  \includeslide[width=\textwidth]{FrameAsignacionVariables} 
\caption{Asignación y operaciones básicas en \texttt{Matlab}\label{FigAsignacionVariables}}
\end{figure}

\begin{figure}
  \includeslide[width=\textwidth]{FrameAsignacionesPython} 
\caption{Asignación y operaciones básicas en \texttt{Python}. Observe la importación del 
        módulo \protect\emph{Numpy} para el tratamiento de Matrices. \protect\label{FigAsignacionVariablesPython}}
\end{figure}

\mode* 

\begin{frame}<presentation>[fragile,label=FrameAsignacionVariables]{Asignación de Variables .m}

\mode<trans>{
  Asignación de una matriz
}
\begin{columns}[T]
\column{0.5\textwidth}

 \vspace{0.5cm}

\hfill \texttt{Asignación de variables}

\vspace{1cm}

\hfill \texttt{Transponer},

\vspace{2cm}

  \hfill \texttt{Referencia a un elemento de la matriz (indexación).}

\vspace{0.8cm}

\hfill \texttt{prompt.}

\column{0.5\textwidth}
\begin{codeblock}
  \verbatiminput{./CODEXAMPLES/ASIGNACION.m}
\end{codeblock}

\end{columns}
\end{frame}

\begin{frame}<presentation>[label=FrameAsignacionesPython]
  \frametitle{Asignacion de Variables .py}
\mode<trans>{
  Asignación de una matriz (version Python)
}
\begin{columns}[T]
\column{0.3\textwidth}

 \vspace{0.5cm}

\flushright \texttt{Asignación de variables}

\vspace{1cm}

\hfill \texttt{Transponer},

\vspace{1cm}

\hfill \texttt{Referencia a un elemento de la matriz (indexación).}

\hfill \texttt{prompt.}

\column{0.7\textwidth}
\begin{codeblock}
  \verbatiminput{./CODEXAMPLES/ASIGNACION.py}
\end{codeblock}

\end{columns}

\end{frame}

\mode<all>

\subsubsection{Indexación de Variables}

\mode<article>

Una vez asignada la matriz, es posible indexar sus componentes. 
Pueden referirse individualmente el elemento de la fila \texttt{i} y la
columna \texttt{j} pidiendo el elemento \texttt{A(i,j)}. Sin 
embargo, es posible realizar operaciones más complejas. Por ejemplo,
puede referirse a un \emph{slice} de la matriz indicando un rango 
de índices en un vector, como se muestra en la
\autoref{FigMatlabIndexacion}. La idea de los \emph{slices} de \emph{arrays} de una o 
más dimensiones persiste en otros lenguajes, y será particularmente
útil más adelante en esta materia, por lo que se sugiere que
verifique su implementación en el lenguaje de programación 
que elija. 

Observe en el ejemplo de la \autoref{FigMatlabIndexacion} para \emph{Python}
la necesidad de usar la función \texttt{ Numpy.ix\_ } para generar todas las 
combinaciones de índices dadas por los índices de las columnas 
\texttt{ ( i\textsubscript{1} , i \textsubscript{2} ) }
y de las columnas \texttt{ ( j\textsubscript{1} , j \textsubscript{2} ) }, 
mientras que en \texttt{Matlab} no es necesario
una función extra. Verifique en su lenguaje de programación cómo debe indexar 
las filas y las columnas para obtener el resultado del ejemplo. 

\begin{figure}
  \includeslide[width=\textwidth]{FrameMatlabIndexacion}
  \caption{
    \protect\label{FigMatlabIndexacion}
  Indexación de Matrices con utilización de listas de índices para obtener un \protect\emph{slice}
  de la matriz
}
\end{figure}

\mode*

\begin{frame}<presentation>[fragile,label=FrameMatlabIndexacion]{Indexación de Variables}

\begin{columns}[T]
\column{0.25\textwidth}
  \vspace{0.5cm}
\flushright \texttt{Rango de filas, todas las columnas}

\column{0.25\textwidth}
  \textbf{Matlab}

\begin{codeblock}
  \verbatiminput{./CODEXAMPLES/Slice1.m}
\end{codeblock}
  \column{0.4\textwidth}
  \textbf{Python}
\begin{codeblock}
%  \verbatiminput{./CODEXAPLES/Slice1.py}
  \verbatiminput{./CODEXAMPLES/Slice1.py}
\end{codeblock}

\end{columns}
  \vspace{0.5cm}
\begin{columns}[T]
\column{0.25\textwidth}
\hfill \texttt{Vector de Índices}

\column{0.25\textwidth}
\begin{codeblock}
  \verbatiminput{./CODEXAMPLES/Slices2.m}
\end{codeblock}

  \column{0.4\textwidth}
  \begin{codeblock}
    \verbatiminput{./CODEXAMPLES/Slices2.py}
  \end{codeblock}
\end{columns}
\end{frame}

\mode<all>
