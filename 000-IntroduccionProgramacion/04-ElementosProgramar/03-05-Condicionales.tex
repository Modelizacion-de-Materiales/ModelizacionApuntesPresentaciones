\subsection{Control de Flujo: Condicionales}

\mode<article>

Otro caso de trabajo frecuente es el de tener 
la necesidad de condicionar la ejecución de 
una pieza de código a la ocurrencia de una 
condición lógica. Para esto se usa la sentencia
\textbf{if}. La estructura más general \textbf{if},
\textbf{else if}, \textbf{else} , \textbf{end} evalúa
una serie de condiciones en forma posicional  y 
excluyente, como se esquematiza en la 
\autoref{FigMatlabIf}. Las sentencias \textbf{else if} y 
\textbf{else} son opcionales. Si la condición 
de argunento en \textbf{if} se cumple, se ejecuta
la serie de comandos hasta \textbf{else} o 
\textbf{end}, lo que se encuentre primero. Al 
terminarse la serie de comandos, no se evalúan 
las condiciones que son argumento de los 
siguientes \textbf{else if}, sino que se
prosigue a partir de end. 

\begin{figure}
  \includeslide[width=\textwidth]{FrameMatlabIf}
  
  \caption{Esquema de la aplicación de los condicionales   para 
  evaluar piezas de código sujetas a condiciones 
  particulares o generales. Notar el uso de los comandos 
  \protect\textbf{continue} y \protect\textbf{break} para forzar la 
  omisión en un lazo o la salida del mismo
  \label{FigMatlabIf}
  }

\end{figure}

Si la condición lógica que es argumento de \textbf{if}
no se cumple, se evalúa 
la condición lógica que es argunmento del 
primer \textbf{else if}, si ésta no se cumple 
se siguen evaluendo los argumentos de las
sentencias \textbf{else if} hasta que se encuentra 
una condición verdadera. En ese caso ocurre lo 
mismo que se explicó antes: cuando se termina
de ejecutar la serie de comandos habilitados
por esta instancia de \textbf{else if}, se
prosigue a partir de \textbf{end}.

La sentencia \textbf{else} da lugar a una serie de
comandos que se ejecutarán solo si ninguna de las 
condiciones lógicas evaluadas es verdadera. 
Podría decirse que marca lo que debe ejecutarse por 
descarte de las condiciones evaluadas. 

En forma general, puede tenerse en cuenta que las 
condiciones sobre las sentencias \textbf{if}, 
\textbf{else if} \emph{ detectan} casos especiales
mientras que \textbf{else} da lugar a la ejecución
sobre el caso más general pero en forma 
excluyente de los casos particulares. Este 
concepto nos será de utilidad en la separación
de las condiciones de contorno sobre un 
recinto de integración para la solución
de ecuaciones diferenciales. 


\mode*

\begin{frame}<presentation>[label=FrameMatlabIf]{Control de flujo: Condicionales} 
\begin{columns}[T]
\column{0.2\textwidth}
\hfill \large\texttt{if}

\column{0.4\textwidth}
\begin{codeblock}
\verbatiminput{./CODEXAMPLES/suma.m}
\end{codeblock}

\column{0.4\textwidth}
\begin{codeblock}
\verbatiminput{./CODEXAMPLES/suma_continue.m}
\end{codeblock}

\end{columns}
\end{frame}

\mode<article>

En la utilización de las distintas herramientas de control de flujo conviene tener a mano
las tablas de verdad. Estas constituyen una serie de operaciones lógicas que pueden servir
para combinar condiciones de ejecución.

\begin{frame}<presentation>[label=FrameTablaVerdad]
  \frametitle{Tabla de verdad:}

\end{frame}
\mode<all>
