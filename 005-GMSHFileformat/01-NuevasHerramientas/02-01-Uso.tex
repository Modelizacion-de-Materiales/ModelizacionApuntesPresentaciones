\mode<article>
\subsection{Uso de herramientas}

Recuerde que todo problema puede separarse en las etapas de 
Preproceso, proceso y Post Proceso. Hasta ahora las únicas 
herramientas usadas fueron papel y lápiz para el pre-proces
y su lenguaje de programación para el Proceso y Post Proceso. 
A partir de ahora se suma la necesidad de usar un software
que nos permita generar los mallados (matriz de nodos y de 
conectividad) ya que la cantidad de elementos y nodos será 
arbitraria para cualquiera de los problemas. Utilizaremos
Gmsh para las etapas de Preproceso y Post Proceso.

Esto implica que deberemos ser capaces de generar la 
comunicación entre nuestro programa de proceso o solver
y la herramienta de pre y post proceso. Con ese 
fin utilizaremos arhivos de texto plano con formato. 
Por lo tanto es necesario que recuerde las sentencias para
lectura y escritura de archivos, introducidas anteriormente en
la Clase Práctica de Introducción a la Programación. Este 
concepto se esquematiza en la \ref{FiguraComunicarHerramientas}
Nuestro programa solver Deberá ser capaz de leer las 
matrices de conectividad y de nodos de las mallas generadas
por Gmsh, y de escribir los resultados en un formato que 
pueda ser usado en esta útlima herramienta para visualizarlos.
Estas relaciones se esquematizan en la Figura 
\ref{FiguraFlujoTrabajo} .
Nuestro Mallador generará la matriz de nodos y 
  la matriz de conectividad a partir de una geometría 
  arbitraria. Toda la información se guarda en archivos 
  de texto que deberemos poder escribir y leer .




\begin{figure}
  \includeslide[width=\textwidth]{FrameComunicarHerramientas}
  \caption{Escquema de la relación entre las herramientas 
  usadas. Usaremos Archivos de texto plano para intercambiar
  información. \label{FiguraComunicarHerramientas} }

\end{figure}

\begin{figure}
  \includeslide[width=\textwidth]{FrameFlujoTrabajo}
  \caption{ Ejemplo del Puente con la Geometría y
  matrices de conectividad y nodos dadas por Gmsh.
  \label{FiguraFlujoTrabajo}
  }
\end{figure}

\mode*
\begin{frame}<presentation>[label=FrameUsoHerramientas]
  \frametitle{Uso De Herramientas}
  \includegraphics[width=\textwidth,page=3,bb=0cm 1cm 28cm 16cm,clip]{./GMSH_fileformat_2019.pdf}
\end{frame}

\begin{frame}<presentation>[label=FrameComunicarHerramientas]
  \frametitle{Comunicación entre Herramientas}
  \includegraphics[width=\textwidth,page=4,bb=0cm 1cm 28cm 16cm,clip]{./GMSH_fileformat_2019.pdf}
\end{frame}

\begin{frame}<presentation>[label=FrameFlujoTrabajo]
  \frametitle{Flujo de trabajo}
  \includegraphics[width=\textwidth,page=5,bb=0cm 1cm 28cm 16cm,clip]{./GMSH_fileformat_2019.pdf}

\end{frame}
\mode<all>
