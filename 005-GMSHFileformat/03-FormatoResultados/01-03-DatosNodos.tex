\mode<article>

Luego de la definición de la geometría y 
las matrices que definen el mallado 
(de nodos y de conectividad) es posible 
adhosar los resultados de los cálculos
hechos por nuestro programa de proceso. 
Típicamente podremos agrupar estos 
resultados como \emph{propiedades de nodos}
y \emph{propiedades de elementos}. Cada dato 
a visualizar tendrá su bloque de 
definición. No hay un orden específico en el cual 
escribir estos bloques, simplemente hay que cuidar
de delimitarlos en forma correcta y que la correlación
entre los datos y los elementos o nodos sea correcta. 

\subsubsection{Datos de Nodos}

Un \emph{bloque de datos de nodo} se delimita por las
etiquetas \texttt{\$NodeData} y \texttt{\$EndNodeData}
como se indica en la Figura \ref{FiguraDatosNodos}.
Debe indicarse el número de títulos que identifican
al bloque, luego deben especificarse los títulos 
entre comillas dobles. 

\begin{figure}

  \includeslide[width=\textwidth]{FrameDatosNodos}
  \caption{Esquema del bloque de datos para los nodos
  \label{FiguraDatosNodos}}

\end{figure}

Luego se indica que sigue una etiquetas real. Esta 
etiqueta indica el valor de tiempo para el cual 
corresponde el bloque de datos. Puede repetirse la 
especificación de bloques de igual título para 
distintos valores de tiempo lo cual permite generar 
una película de una propiedad con variación temporal.

Sigue a esto el número de etiquetas enteras. Estas 
especifican el índice temporar, la dimensionalidad del
dato (escalar =1, vectorial =3, tensorial =6). 

Por último, se indica el número de datos a escribir.
en general hay un dato por cada nodo. 

Deben escribirse los datos especificando el nodo al
cual corresponde. 

\subsubsection{Datos para los Elementos}

La estructura del bloque de datos para los elementos
es similar a la correspondiente para los nodos. 
En esta caso el bloque estará delimitado por las
etiquetas \texttt{\$ElementData} y 
\texttt{\$EndElementData} como se esquematiza en la 
Figura \ref{FiguraDatosElementos} 

\begin{figure}

  \includeslide[width=\textwidth]{FrameDatosElemento}
  \caption{Esquema del bloque de datos para los 
  Elementos\label{FiguraDatosElementos} }

\end{figure}

\mode* 

\begin{frame}<presentation>[label=FrameDatosNodos]
  \frametitle{Datos de Nodos}
  \includegraphics[width=\textwidth,page=10,
  bb=0cm 0cm 28cm 16.8cm,clip]{./Libreoffice/GMSH_fileformat.pdf}



\end{frame}


\begin{frame}<presentation>[label=FrameDatosElemento]
  \frametitle{Datos para Elementos}
  \includegraphics[width=\textwidth,page=11,
  bb=0cm 0cm 28cm 16.5cm,clip]{./Libreoffice/GMSH_fileformat.pdf}

\end{frame}

\mode<all>
