\mode<all>
% Incluyo los paquetes necesarios.
% para no tener problemas con acentos etc.
\usepackage[utf8]{inputenc}
% en español
\usepackage[spanish]{babel}
%matemática
\usepackage{amsmath}
% este no se si hace falta pero por las dudas
\usepackage{graphicx}
% para incluir peliculas
\usepackage{multimedia}
% para usar segunda pantalla
\usepackage{pgfpages}
\usepackage{pgf}
% para hacer dibujitos
\usepackage{tikz}
\usetikzlibrary[automata,calc,arrows,decorations.pathmorphing,backgrounds,shapes,
patterns,positioning,fit,petri,overlay-beamer-styles]
\tikzstyle{every picture}+=[remember picture]
%recuadros sencillos
%\usepackage{tcolorbox}
% enumeradores intercambiables
\usepackage{enumerate}
% para subtitulos en figuras
\usepackage{subcaption}
%listings for code input
\usepackage{listings}
%verbatim input from file!
\usepackage{verbatim}
\usepackage{fancyvrb}
% para modificar los encabezados y pies de página.
%\usepackage{fancyhdr}
%\pagestyle{fancy}
\usepackage{standalone}

\definecolor{codegreen}{rgb}{0,0.6,0}
\definecolor{codegray}{rgb}{0.5,0.5,0.5}
\definecolor{codepurple}{rgb}{0.58,0,0.82}
\definecolor{backcolour}{rgb}{0.95,0.95,0.92}

\lstdefinestyle{codeblock}{
    backgroundcolor=\color{backcolour},   
    commentstyle=\color{codegreen},
    keywordstyle=\color{magenta},
    numberstyle=\tiny\color{codegray},
    stringstyle=\color{codepurple},
    basicstyle=\ttfamily\footnotesize,
    breakatwhitespace=false,         
    breaklines=true,                 
    captionpos=b,                    
    keepspaces=true,                 
    numbers=left,                    
    numbersep=5pt,                  
    showspaces=false,                
    showstringspaces=false,
    showtabs=false,                  
    tabsize=2
}


%\mode<presentation>{
% para ver las notas con la presentacion.  
%\setbeameroption{hide notes}
%para dejar las notas en la segunda patnalla
%}

% incluyo los beamercolors
\include{./PREAMBLE/BEAMERCOLORS}

%incluyo el tema y modificaciones
\include{./PREAMBLE/BEAMERTHEME}

% modifico los temas
\include{./PREAMBLE/THEMES}
% defino el template para la diapositiva del título

%%%%%%%%%%%%%%%%%%%%%%%%%%%%%%%%
% Defino la Clase
%%%%%%%%%%%%%%%%%%%%%%%%%%%%%%%%
\subtitle[Modelización 2020]{ Modelización de Propiedades y Procesos 2020 }
\author{
  Ruben Weht\inst{1,2} 
  \and Silvio Terlisky\inst{1,3}
  \and Pablo Gargano\inst{1,3} 
  \and Mariano Forti\inst{1,3} 
}
\institute{
  \inst{1}Instituto de Tecnología Prof. Jorge Sabato
  \and
  \inst{2}Fisica del Sólido, Edificio TANDAR,
  \and
  \inst{3} Gerencia Materiales
}

\mode<presentation>{\date{\scriptsize\href{mailto:model.sabato@gmail.com}{model.sabato@gmail.com}}}

\mode<article>{
  \date{
    \small
  \textsuperscript{1}Instituto de Tecnología Prof. Jorge Sabato\\
  \textsuperscript{2}Fisica del Sólido\\
  \textsuperscript{3}Gerencia Materiales\\
  \href{mailto:model.sabato@gmail.com}{model.sabato@gmail.com}
}

%defino los encabezados y pies de págna para
% dodo el documento en función de la materia y la
% clase.
%\fancyhead[L]{\tiny Modelización de Materiales 2019}
%\fancyhead[R]{\tiny \leftmark}
}

\title{
  \mode<article>{
% \includegraphics[height=1cm]{./PREAMBLE/logo-isabt25.png}
\includegraphics[height=1cm]{./PREAMBLE/Beninson.jpeg}
\hfill
\includegraphics[height=1cm]{./PREAMBLE/ISOLOGOCNEA.png}
\hfill
\includegraphics[height=1cm]{./PREAMBLE/logo-unsam.png}
\\}  Apéndice: Instalación de Python más fácil que nunca}
\subject{Apéndice: Instalación de Python más fácil que nunca}
\keywords{Python, Modelizacion 2021, Programación}
%linea para fbox
\setlength\fboxsep{0pt}
\setlength\fboxrule{5pt}
% Inicia el documento.
\begin{document}

% Título de la clase. 
\mode<presentation>{
\begin{frame}[plain]
\titlepage
\end{frame}
}

%\mode<article>{
%\maketitle
%}

\begin{frame}<presentation>[label=FrameWinPython]
  \frametitle{https://winpython.github.io/}
  \center
  \href{https://winpython.github.io/}{ 
  \includegraphics[width=0.5\textwidth]{winpython_title.png}
  }

  \includegraphics[width=\textwidth]{winpython_launchers.png}

  \href{https://sourceforge.net/projects/winpython/files/latest/download}{Descargar últuma version desde SourceForge}

  \tiny https://sourceforge.net/projects/winpython/files/latest/download
\end{frame}

\begin{frame}<presentation>[label=FrameEsperar]
  \frametitle{Esperar}
\center
  \includegraphics[height=0.9\textheight]{mate.jpg}

\end{frame}

\begin{frame}<presentation>[label=FrameExctract]
  \frametitle{Extraer Archivos}
 

  \begin{tikzpicture}
    \node<1-> {\fbox{\includegraphics[width=0.5\textwidth]{Screenshots/RunAsAdministrator.png}}};

    \node<2-> at (3,0) {\fbox{\includegraphics[width=0.5\textwidth]{Screenshots/GiveAccess.png}}};

    \node<3-> at (7,-1) {\fbox{\includegraphics[width=0.5\textwidth]{Screenshots/ChangePath.png}}};

  \end{tikzpicture}

\end{frame}

\begin{frame}<presentation>[label=FrameEsperarII]
  \frametitle{Esperar (II)}
\center
  \href{https://www.youtube.com/playlist?list=PLQkH8HQf-uqoNlgHH9g0eBzeIQbvKiilH}{
    \includegraphics[height=\textheight]{oz-no-mahotsukai-episode-17_521.png}
  }
\end{frame}

\section{Probar}
\begin{frame}<presentation>[label=FrameProbar]
  \frametitle{Probamos}

  \begin{tikzpicture}
    \node<1-> at (0,0) {\fbox{\includegraphics[width=0.5\textwidth]{Screenshots/PythonPrompt.png}}};

    \node<2-> at (3,-1) {\fbox{\includegraphics[width=0.5\textwidth]{Screenshots/JugamosUnRato.png}}};

    \node<3-> at (6,-1) {\fbox{\includegraphics[width=0.5\textwidth]{Screenshots/UnGrafico.png}}};

  \end{tikzpicture}


\end{frame}

\section{Configuracion}

\begin{frame}<presentation>[label=FrameConfigurar]
  \frametitle{Mínima configuracion}

  \begin{tikzpicture}

    \node at (0,0) {\fbox{\includegraphics[width=0.5\textwidth]{Screenshots/PythonPrompt.png}}};

    \node at (5,0) {\fbox{\includegraphics[width=0.6\textwidth]{Screenshots/Configurar.png}}};

  \end{tikzpicture}
\end{frame}

\section{Usamos Jupyter Lab}

\begin{frame}<presentation>[label=FrameJupyterLab]
  \frametitle{Abrimos Jupyter Lab desde una carpeta}

  \center

  \begin{columns}
    \column{0.6\textwidth}
    \includegraphics[height=0.7\textheight]{Screenshots/ClickIzquierdo.png}
    \column{0.4\textwidth}

    \begin{itemize}
	\item Creamos una carpeta de trabajo y navegamos hasta allí

	\item click izquierdo
	
	\item click en \emph{Jupyter Lab Here}

    \end{itemize}

  \end{columns}
\end{frame}

\begin{frame}<presentation>[label=FrameEnjoy]
  \frametitle{¡A trabajar !}

  \center
  \includegraphics[width=0.6\textwidth]{Screenshots/PlayHappy.png}

\end{frame}

\end{document}
