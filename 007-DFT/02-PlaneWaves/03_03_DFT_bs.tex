\mode*

\begin{frame}<presentation>[label=FrameBandStructure]
  \frametitle{Cristales}

      Bloch theorem

%	\begin{equation*}
	$$ \varphi_{\mu, k} = \sum_{G} C_{\mu, G, k}  e^{i (\vec{G} + \vec{k} )\cdot \vec{r}} $$  \tikz[remember picture, overlay] \node (A) {};
	%\end{equation*}

	$$ \left ( \nabla^2 + V_{Ne}[\rho] +V_{ee}[\rho] +V_{XC}[\rho] \right ) \varphi _\mu = \varepsilon _{\mu} \varphi_{\mu} $$

Uno termina resolviendo un sistema lineal para cada punto $\vec{r}$ :

    $$ \mathbf{C} _{\mu, \vec {k} + \vec{G} } E[\rho] \varphi _{\mu, \vec{k} + \vec{G}}(\vec {r}) = \varepsilon _{\mu, \vec{k} + \vec{G}} \varphi _{\mu, \vec{k} + \vec{G}}(\vec {r}) $$


\end{frame}

\begin{frame}<presentation>[label=FrameNmerosCunticos]
  \frametitle{Números Cuánticos}
%
$$ \mathbf{C} _{\mu, \vec {k} + \vec{G} } E[\rho] \varphi _{\mu, \vec{k} + \vec{G}}(\vec {r}) = \varepsilon _{\mu, \vec{k} + \vec{G}} \varphi _{\mu, \vec{k} + \vec{G}}(\vec {r}) $$


   \begin{itemize}
      \item $\vec{k} \in BZ $ Red recíproca del sistema ! (por teorema de Bloch). 

      \item $|\vec{k} + \vec{G}| <= K_{max} $ para que la suma sea finita: energía de corte!

      \item $\mu $: niveles de energía o bandas de un cristal

      \item $\varepsilon _{\mu, \vec{k} }$ : estructura de bandas !

    \end{itemize}
%
%
\end{frame}
