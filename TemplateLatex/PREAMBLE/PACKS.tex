% para no tener problemas con acentos etc.
\usepackage[utf8]{inputenc}
% en español
\usepackage[spanish]{babel}
%matemática
\usepackage{amsmath}
% este no se si hace falta pero por las dudas
\usepackage{graphicx}
% para incluir peliculas
\usepackage{multimedia}
% para usar segunda pantalla
\usepackage{pgfpages}
\usepackage{pgf}
% para hacer dibujitos
\usepackage{tikz}
\usetikzlibrary[automata,calc,arrows,decorations.pathmorphing,backgrounds,shapes,
patterns,positioning,fit,petri,overlay-beamer-styles]
\tikzstyle{every picture}+=[remember picture]
%recuadros sencillos
%\usepackage{tcolorbox}
% enumeradores intercambiables
\usepackage{enumerate}
% para subtitulos en figuras
\usepackage{subcaption}
%listings for code input
\usepackage{listings}
%verbatim input from file!
\usepackage{verbatim}
\usepackage{fancyvrb}
% para modificar los encabezados y pies de página.
%\usepackage{fancyhdr}
%\pagestyle{fancy}
\usepackage{standalone}

\definecolor{codegreen}{rgb}{0,0.6,0}
\definecolor{codegray}{rgb}{0.5,0.5,0.5}
\definecolor{codepurple}{rgb}{0.58,0,0.82}
\definecolor{backcolour}{rgb}{0.95,0.95,0.92}

\lstdefinestyle{codeblock}{
    backgroundcolor=\color{backcolour},   
    commentstyle=\color{codegreen},
    keywordstyle=\color{magenta},
    numberstyle=\tiny\color{codegray},
    stringstyle=\color{codepurple},
    basicstyle=\ttfamily\footnotesize,
    breakatwhitespace=false,         
    breaklines=true,                 
    captionpos=b,                    
    keepspaces=true,                 
    numbers=left,                    
    numbersep=5pt,                  
    showspaces=false,                
    showstringspaces=false,
    showtabs=false,                  
    tabsize=2
}

