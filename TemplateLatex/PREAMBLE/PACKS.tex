% para no tener problemas con acentos etc.
\usepackage[utf8]{inputenc}
% en español
\usepackage[spanish]{babel}
%matemática
\usepackage{amsmath}
% este no se si hace falta pero por las dudas
\usepackage{graphicx}
% para incluir peliculas
\usepackage{multimedia}
% para usar segunda pantalla
\usepackage{pgfpages}
\usepackage{pgf}
% para hacer dibujitos
\usepackage{tikz}
\usetikzlibrary[automata,calc,arrows,decorations.pathmorphing,backgrounds,
patterns,positioning,fit,petri,overlay-beamer-styles]
\tikzstyle{every picture}+=[remember picture]
%recuadros sencillos
%\usepackage{tcolorbox}
% enumeradores intercambiables
\usepackage{enumerate}
% para subtitulos en figuras
\usepackage{subcaption}

%verbatim input from file!
\usepackage{verbatim}
% para modificar los encabezados y pies de página.
%\usepackage{fancyhdr}
%\pagestyle{fancy}
\usepackage{standalone}
\usepackage[pdfauthor="Mariano Forti, Ruben Weht", 
            pdftitle="Modelizacion de Materiales y Procesos",
	    spanish, hidelinks]
	    {hypperref}
