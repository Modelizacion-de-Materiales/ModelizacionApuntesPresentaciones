\mode<article>

El primer paso en este sentido consiste en discretizar 
el dominio de la chapa en un conjunto de puntos regularmente 
distribuídos. Al tratarse de un recinto cuadrado, este paso es trivial.

Para formar la grilla dividimos los lados sobre 
los ejes $x$ e $y$ del cuadrado en $N_x-1$ y $N_y-1$
segmentos iguales, determinando $N_x$ y $N_y$ puntos
equidistantes sobre cada uno. De las intersecciones
de las líneas paralelas a los lados que se 
extienden de los puntos formados se genera la grilla 
a usar, como se muestra en la Figura \ref{FiguraDiscretizacion}.
Puede verse inmediatamente que la cantidad de nodos
total de la grila es $N_x N_y$

\begin{figure}
  \includeslide[width=\textwidth]{FrameDiscretizacion}
  \caption{Grilla de discretización del dominio de la chapa\label{FiguraDiscretizacion}}
\end{figure}

%Supongamos por un instante que conocemos la distribución
%de temperaturas en la chapa.
Para poder asignar una temperatura a cada nodo de la grilla, 
debemos tomar una convención para numerarlos. La primera
intuición tiene que ver con una numeración de índices según 
las dos direcciones cartesianas $x$ e $y$. Necesitamos 
entonces dos índices $i$ y $j$, de manera que las 
temperaturas $T(x_i,y_j)$ para los nodos de la grilla
pueden representarse en forma matricial $T_{i,j}$.

Sin embargo, por razones que pronto quedarán claras, es 
conveniente tomar una numeración de un único índice $k$
para cada nodo.  
Con alguna arbitrariedad, puede tomarse una numeración
correlativa asignando $k=1$ al nodo del 
el vértice inferior izquierdo, hacia la 
derecha. En cada ``fin de línea'' se toma el índice 
siguiente para el primer nodo de la línea superior. 
Con esta convención, puede verse que los vértices 
de la chapa quedan identificados en función de las dimensiones
de la grilla elegida

\begin{equation}
  \begin{split}
    \text{Vertice inferior izquierdo:   } & k=1\\
    \text{Vertice inferior derecho:     } & k=N_x \\
    \text{Vertice superior izquierdo:   } & k=N_xN_y-N_x+1\\
    \text{Vertice superior derecho:     } & k=N_xN_y
  \end{split}
  \label{EqEcuacionesNumeracionVerticies}
\end{equation}

Ahora, las temperaturas de la barra 
pueden representarse en un \textbf{vector} 

\begin{equation} \label{EqEcuacionVectorT}
  \vec{T} = 
  \begin{pmatrix}
    T_1 \\ T_2 \\ \vdots \\ T_{Nx Ny}
  \end{pmatrix}
\end{equation}

Es conveniente encontrar una relación entre los sistemas
de índices. como $1 \leq i \leq N_x $ ; $1 \leq j \leq  N_y$, puede verse
fácilmente que dicha relación es

\begin{equation}\label{EqEcuacionRelacionIndices}
  \begin{aligned}
  k &= i + (j-1)N_x \\
  j &= floor \bigg( \frac{k-1}{N_x} +1  \bigg)\\
  i &= k - (j-1)N_x
  \end{aligned}
\end{equation}
donde la función $floor$ indica que debe tomarse la parte
entera inferior de su argumento. Por otro lado, las relaciones
de la ecuación \ref{EqEcuacionRelacionIndices} permite
relacionar ambas convenciones de índices en forma biyectiva, 
de manera que podamos relacionarlo
con la numeración de dos índices $i,j$ en forma unívoca.


Según el método de las diferencias fnitas, debemos
usar las versiones \emph{discretas} de las derivadas
segundas en la ecuación \ref{EqEcuacionPoisson}. 
Puede consultar la teórica para rescatar estas 
definiciones, pero aquí razonaremos en base a las
mismas, que las derivadas segundas pueden calcularse
a partir de las derivadas primeras. 

Para empezar por el problema más sencillo, busquemos la
segunda derivada parcial respecto de x como un cociente 
incremental de las derivadas primeras. 
Dado un nodo de 
numeración $k$ \emph{que no se encuentre en ningún borde
de la chapa}, notemos según nuestra convención el punto
siguiente hacia las $ x $ positivas es el número $k+1$, mientras
que el nodo hacia las $ x $ negativas es el $k-1$. Podemos
definir entonces las derivadas primera y segunda en el
nodo $k$-ésimo como 

\begin{equation}\label{EqEcuacionDerivadaX}
    \begin{aligned}
    \frac{\partial T}{\partial x} \Bigr\rvert _k &=
			   \frac{T_{k}-T_{k-1}}{\Delta x} &= dT_k  \\
    \frac{\partial^2 T}{\partial x^2} \Bigr\rvert _k &=
    \frac{dT_{k+1}- dT_k}{\Delta x}                  &= 
			   \frac{1}{\Delta x} \big(
			   \frac{T_{k+1}-T_k}{\Delta x}
			   -
			   \frac{T_{k}-T_{k-1} }{\Delta x}
			   \big)\\
    \Rightarrow  &= 
    \frac{T_{k+1} - 2T_k +T_{k-1} }{\Delta x ^2}
  \end{aligned}
\end{equation}

%  \label{EqDerivadaDiscretaX}
%\end{multline}

Para poder calcular la derivada segunda en la otra dirección 
debemos notar que es necesario tomar cocientes incrementales
de la temperatura hacia arriba y hacia abajo. Debe notar que 
para un nodo de índice $k$, el nodo de \emph{arriva} tiene 
el orden $k+N_x$. Haga el ejercicio de obtener la derivada 
segunda respecto de y

\begin{equation}\label{EqEcuacionDerivadaY}
    \frac{\partial^2 T}{\partial y^2} \Bigr\rvert _k 
    =
    \frac{T_{k+N_x} - 2T_k +T_{k-N_y} }{\Delta y ^2}
\end{equation}



\mode*

\begin{frame}<presentation>[label=FrameDiscretizacion]
  \frametitle{Discretización del Problema}
  \begin{columns}
    \column{0.6\textwidth}
    \includegraphics[width=\textwidth]{./01-Problema/01-Figura-Grilla.pdf}
%    \input{./01-Problema/01-Figura-Grilla.tikz}
    \column{0.5\textwidth}
\centering
$\frac{T_{k+1} - 2T_k +T_{k-1} }
  {\Delta x ^2} +
 \frac{ T_{k+N_x} - 2T_k +T_{k-N_y} }{ \Delta y^2 } =0 $

    \vspace{1cm}
    $ \vec{T}= \begin{pmatrix}
    T_1 \\ T_2 \\ \vdots \\ T_{Nx Ny}
     \end{pmatrix}$ 

  \end{columns}
\end{frame}

\mode<all>
