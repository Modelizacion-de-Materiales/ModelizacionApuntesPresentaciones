\mode<article>

Vamos a encarar el problema presentado en la Figura \ref{FiuguraPresentacionProblema}. 
se tiene una chapa dispuesta bajo condiciones de contorno tales que sus lados se 
encuentran a temperaturas fijas  $T_A$, $T_B$, $T_C$, $T_D$.

Queremos obtener la distribución de temperatura dentro 
del recinto de la chapa, en condiciones estacionarias, 
es decir, cuando las temperaturas de no varían en ningún 
punto del recinto. Por lo tanto, podemos elegir el 
modelo físico a resolver. La ecuación diferencial 
que rige el problema es la ecuación de Poison, que es 
la ecuación de transferencia térmica homogénea. 

  \begin{equation}
    \frac{\partial ^2 T_{(x,y)}}{\partial x ^2}
    +
    \frac{\partial ^2 T_{(x,y)} }{\partial y ^2}
    =0
    \label{EqEcuacionPoisson}
  \end{equation}

Debido a que los contornos del recinto donde debemos
resolver la ecuación \ref{EqEcuacionPoisson} son los adecuados,
elegimos utilizar el método de \textbf{Diferencias Finitas}. 


\begin{figure}
  \includeslide[width=\textwidth]{FrameProblema}
  \caption{Presentación del Problema. Chapa bidimensional 
  de conductividad térmica finita, cuyos bordes se encuentran 
  a temperaturas constantes finitas $T_A$, $T_B$, $T_C$, $T_D$}
  \label{FiuguraPresentacionProblema}.
\end{figure}

\mode<all>

\mode*

\begin{frame}<presentation:1|article:0>[label=FrameProblema]
  \frametitle<1>{Presentación del Problema}
  \frametitle<2>{Bordes a temperatura Fija}
%  \begin{tikzpicture}[overlay]
%    \draw [draw,thick,pattern=north east lines] (4.9,-3.1) rectangle (9.1,1.1);
%    \draw [draw,line width=2pt,anchor=base,fill=white] (5,-3) rectangle (9,1);
%    \draw [<->, >=latex, line width = 2pt ]
%    (2,-1) node [anchor=south] { $y_j$ }  -- (2,-3) -- (4,-3) node [anchor=south] {$x_i$};
%  \end{tikzpicture}
  \tikz[overlay] \node at (6,-1) {
    \includegraphics[width=9 cm]{./01-Problema/Figura01-01-Chapa.pdf}
  };

%  \begin{equation}
$$  \frac{\partial ^2 T_{(x,y)}}{\partial x ^2}
    +
    \frac{\partial ^2 T_{(x,y)} }{\partial y ^2}
    =0 $$
%  \end{equation}

\end{frame}

\mode<all>
