\mode<article>

Hasta aquí resolvimos el problema con condiciones de contorno de 
temperaturas fijas. Si queremos incorporar una condición de 
contorno para derivadas (flujos) sobre alguno de los bordes
debemos tener en cuenta el siguiente truco. Consideremos como 
ejemplo el flujo entrante hacia la chapa sobre el borde 
izquierdo. En un punto de ese borde, podemos aproximar el flujo entrante
de calor a una cantidad proporcional a la derivada primera,

\begin{equation}\label{EqFluxPartialT}
  Q_{x} \propto  \dfrac{\partial T }{\partial x} 
\end{equation}

\mode*

\begin{frame}<presentation>[label=FrameGrillaFlujoDado]
  \frametitle{Borde Izquierdo con Flujo}

  \includegraphics[width=\textheight,page=16]{./clase-lo/EDP2019.pdf}

\end{frame}

\mode<all>
