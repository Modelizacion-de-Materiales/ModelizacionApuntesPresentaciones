\mode<article>

Sabiendo los valores de los coeficientes de la matriz que 
representa a nuestra ecuación diferencial, podemos trazar la 
lógica de nuestro modelo computacional.

Cuando inicie el estudio del problema, no use tamaños de grillas 
grandes. Utilice la mínima grilla posible, por ejemplo
$N_x = N_y = 3$ para poder interpretar los resultados 
en forma legible. 

\begin{figure}
  \includeslide[width=\textwidth]{FrameLogica}
  \caption{Esquema lógico del problema para el caso con temperatura fija en los bordes.}
  \label{FiuguraLogica}.
\end{figure}

\mode*

\begin{frame}<presentation>[label=FrameLogicaEnumerate]
  \frametitle{Lógica del programa}
  
\begin{enumerate}
    \item<+-> Recorrer todos los valores posibles para el índice k, es decir los nodos
    \item<+-> decidir si el nodo está en un borde o en el interior de la chapa
    \item<+-> llenar la fila correspondiente de la matriz y al vector $\vec{b}$
      en consecuencia.
    \item<+-> resolver el sistema lineal
    \item<+-> Graficar !
\end{enumerate}

\end{frame}

\begin{frame}<presentation>[label=FrameLogica]
  \frametitle{Lógica del modelo}
  \begin{tikzpicture}[every node/.style={
      rectangle, minimum size=1cm, draw, thick,
      align=center, font=\ttfamily
    },
    every path/.style={ >=latex },
    node distance=1.5cm and 1.5cm]

    % column 1
    \node (start) {Leer Condiciones\\de Contorno };
    \node[below of=start] (fork) {$1\leq k \leq N_x N_y $};
    \node [below of=fork] (vertex) {$k \in k_A,k_B, k_C,k_D$ ?};

    %column 2
    \node [right=of start] (matrix) {Armar la\\matriz};
    \node [below of=matrix] (sizes) {$N_x , N_y$};
    \node [below of=sizes](matcc) {
      \scalebox{0.5}{$ M[ k, :] = [\dotsi , 0,  1,0,\dotsi] $ } };
    \node [below of=matcc] (eqint) { \scalebox{0.5}{
      $M[k,[k-N_x,k-1,k,k+1,k+N_x]]=[ \beta ^2 , 1 , 2\big(1+\beta^2\big),  1 \beta^2]$
    }
    };

    % column 3
    \node [right=of matrix] (solve) {resolver el\\sistema};

    %column 4
    \node [right=of solve] (graph) {graficar};

    %row 1
    \draw [->] (start) -- (matrix);
    \draw [->] (matrix) -- (solve);
    \draw [->] (solve) -- (graph);
    \draw[->] (vertex) -- (matcc) node [midway,above,draw=none] {Si};
    \draw[->] (vertex.south) |- (eqint.west) node [midway,left,draw=none] {No};
    
  \end{tikzpicture}
\end{frame}

\mode<all>
