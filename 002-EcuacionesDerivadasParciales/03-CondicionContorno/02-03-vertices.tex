\mode<article>

El caso de los vértices debe ser tratado en forma separada. 
esto se debe a que según la definición de los bordes en 
la ecuación \ref{EqNumeracionBordes}, los nodos en los 
vértices pueden pertenecer a dos bordes simultáneamente.
Por ejemplo el nodo $k=1$ pertenece a $k_A$ y a $K_C$. 

Pueden tomarse varias estrategias para atacar estos 
problemas. Podemos tomar preferencia por alguno de 
los bordes. Entonces la lógica que adoptaremos 
debe evaluar primero si los nodos están en los
bordes según la preferencia dada. 

Otra alternativa es que la lógica de la solución 
evalúe primero si el nodo está en algún borde, se 
le asigne alguna condición de contorno específica.
Una vez descartada la pertenencia a todos los vértices,
evaluar la pertecia a los bordes. 
En este úlimo caso es sencillo asignar por ejemplo
el promedio de las temperaturas de los bordes 
subyacentes al vértice, por ejemplo.

\mode*
\begin{frame}<presentation:0>[label=FrameEmpty1]
  \frametitle{}
\end{frame}
\mode<all>
