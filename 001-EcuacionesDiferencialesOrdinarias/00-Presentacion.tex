\subsection{Ecuación Diferencial}

Nos concentraremos en las ecuaciones diferenciales de la forma

\begin{equation}\label{EqnEDO}
  y' = f(x, y)
\end{equation}

\noindent lo que nos dice que la derivada de la funicón depende tanto del valor de la misma función
$y$ como de la variable independiente $x$.

Para resolver esta ecuación en forma numérica, debemos discretizar la derivada, 

\begin{equation}\label{EqnDeltayDeltax}
  \begin{aligned}
    y' &= \drac{y_{i+1} - y_i}{x_{i+1} - x_i}\\
    y' &= \drac{y_{i+1} - y_i}{dx}
  \end{aligned}
\end{equation}


\subsubsection{Problemas de condición inicial}

La ecuación \ref{EqnDeltayDeltax}  hace evidente la naturaleza iterativa de la solución numérica. 
Necesitamos una condición inicial, es decir el valor de la función $y$ para algún valor $x_0$ 
que luego propagaremos a todos los valores posibles de $x$ con algúm método adecuado.

\begin{equation}\label{EqnCondicionInicail}
  y_0 = y(x_0)
\end{equation}

Este tipo de razonamiento generalmente se aplica cuando la variable independiente es el tiempo, 
y la condición inicial se propaga utilizando la \autoref{

\subsection{Método de Euler}

Podemos pensar en hacer una sustitución de la expresión de la ecuación \ref{EqnDeltayDeltax} en la
\autoref{EqnEDO1} para despejar el el valor de la función a un 
