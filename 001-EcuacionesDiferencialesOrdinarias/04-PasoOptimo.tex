\mode<article>

Una pregunta que estamos en condiciones de responder es cuáles son los parámetros 
de cáclulo más adecuados. En este caso podemos explorar distintos tamaños de paso
e intentar tomar conclusiones.

Por ejemplo midiendo el error como el valor de la velocidad al final de la caída 
respecto de la solución teórica, y variando el tamaño de paso para barrer varios 
órdenes de magnitud, podemos construir el gráfico de la \autoref{FiguraErrorEsfuerzo}.
En esta medida tomamos el esfuerzo de cálculo como la cantidad de veces que se evalúa la 
función $F$ para cada solución. Tenemos entonces que para el método de Runge-Kutta de orden 
4 la función se evalúa cuatro veces en cada iteración, pero el error de truncamiento 
cometido disminuye rápidamente hasta competir con el error de redondeo.

\begin{figure}[H]
  \includeslide[width=\textwidth]{FrameErroresEsfuerzo}
  \caption{\protect\label{FiguraErrorEsfuerzo} 
  Error cometido por cada método en función del esfuerzo de cálculo, que es proporcional 
  al tamaño de paso.}
\end{figure}

\mode*

\begin{frame}<presentation>[label=FrameErroresEsfuerzo]
  \frametitle{Buscando el paso óptimo}
  \center
  \includegraphics[height=7cm, trim=0cm 6cm 0cm 6cm, clip]{RESULTS/errores.pdf}
\end{frame}

\mode<all>
