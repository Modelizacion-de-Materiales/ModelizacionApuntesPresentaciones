\mode<article>

Para resolver el problema vamos a implementar la iteración de la \autoref{EqnItera}
pero con la aproximación a la derivada del método elegido. 
Las funciones \texttt{pasoEU} y \texttt{pasoRK} que pueden verse
en la \autoref{FigPasoRK}.

\begin{figure}[H]
  \includeslide[width=\textwidth]{FramePasoEU}
  \caption{\protect\label{FigPasoEu} 
  Paso de propagación para el método de Euler. se provee una función que devuelve los valores en el paso
  siguiente a partir del paso previo.}
\end{figure}

\begin{figure}
  \includeslide[width=\textwidth]{FramePasoRK}

  \includeslide[width=\textwidth]{FramePasoRKCode}

  \caption{\protect\label{FigPasoRK}
  Paso de propagación para el método de Runge-Kutta
  }
\end{figure}

\mode*

\begin{frame}<presentation>[label=FramePasoEU]
  \frametitle{Paso para Euler}
    \center
    \begin{tikzpicture}[every path/.style={line width=2pt}]
      \draw[->] (-8,0) -- (-1, 0) node[midway, above] {$[x_{i-1}, v_{i-1}], t_{i-1}$, \texttt{@dfparac} };
      %\path[draw] (-1, -0.5) rectangle (1, 0.5) node[midway] (funcion) {\texttt{pasoEU}};
      \node[draw, minimum width=2cm, minimum height=1cm] (funcion) {\texttt{pasoEU}};

      \draw[->] (1,0) -- (5, 0) node[midway, above] {$[x_{i}, v_{i}], t_{i}$ };
      \node[below=1cm of funcion,draw, rounded corners] (detail)  {
	$ Y_i = Y_{i-1} + dt F \Big( t_{i-1}, [x_{i-1}, v_{i-1}] \Big)$
	};
	\draw (funcion.south) -- (detail.north);
    \end{tikzpicture}

    \begin{codeblock}
      \VerbatimInput[frame=lines, firstline=23]{CODES/pasoEU.m}
    \end{codeblock}

\end{frame}

\begin{frame}<presentation>[label=FramePasoRK]
  \frametitle{Paso para Runge-Kutta}
  \center
      \begin{tikzpicture}[every path/.style={line width=2pt}]
	\draw[->] (-8,0) -- (-1, 0) node[midway, above] {$[x_{i-1}, v_{i-1}], t_{i-1}$, \texttt{@dfparac} };
	%\path[draw] (-1, -0.5) rectangle (1, 0.5) node[midway] (funcion) {\texttt{pasoEU}};
	\node[draw, minimum width=2cm, minimum height=1cm] (funcion) {\texttt{pasoRK}};
	\draw[->] (1,0) -- (5, 0) node[midway, above] {$[x_{i}, v_{i}], t_{i}$ };
	\node[below=1cm of funcion,draw, rounded corners] (detail)  {
	  $ 
	    \begin{aligned}
	      k_1 &= F (x_i, X_i, Y_i)\\
	      k_2 &= F \Big( X_i + \frac{1}{2} \Delta x, Y_i + \frac{1}{2} k_1 \Delta x \Big)\\
	      k_3 &= F \Big( X_i + \frac{1}{2} \Delta x, y_i + \frac{1}{2} k_2 \Delta x \Big)\\
	      k_4 &= F \Big( X_i \Delta x, Y_i + k_3 \Delta x \Big) \\
	      Y_{i+1} &= Y_i+\frac{1}{6} \Big( k_1 + 2 k_2 + 2 k_3 + k_4 \Big) \Delta x
	    \end{aligned}
	  $
	  };
	  \draw (funcion.south) -- (detail.north);
      \end{tikzpicture}
\end{frame}

\begin{frame}<presentation>[label=FramePasoRKCode]
  \frametitle{Código Paso Runge Kutta}
  \begin{codeblock}
    \VerbatimInput[firstline=23, frame=lines]{CODES/pasoRK4.m}

  \end{codeblock}

\end{frame}
\mode<all>
