Para resolver el problema vamos a implementar la iteración de la \autoref{Eqn}
pero con la aproximación a la derivada del método elegido. 
Las funciones \texttt{pasoEU} y \texttt{pasoRK} que pueden verse
en las  \autoref{FigPasoEU} u \autoref{FigPasoRK} 

\mode*

\begin{frame}[label=FramePasoEU]
  \frametitle<presentation>{Paso para Euler}
  \begin{figure}
    \center
    \begin{tikzpicture}[every path/.style={line width=2pt}]
      \draw[->] (-5,0) -- (-1, 0) node[midway, above] {$[x_{i-1}, v_{i-1}], t_{i-1}$ };
      %\path[draw] (-1, -0.5) rectangle (1, 0.5) node[midway] (funcion) {\texttt{pasoEU}};
      \node[draw] (funcion) {\texttt{pasoEU}};

      \draw[->] (1,0) -- (5, 0) node[midway, above] {$[x_{i}, v_{i}], t_{i}$ };
      \node[below=1cm of funcion,draw, rounded corners] (detail)  {
	$ Y_i = Y_{i-1} + dt F \Big( t_{i-1}, [x_{i-1}, v_{i-1}] \Big)$
	};
	\draw (funcion.south) -- (detail.north);
    \end{tikzpicture}
    \mode<article>{
      \protect\label{FigPasoEu}
    }
  \end{figure}
\end{frame}


\begin{figure}
  \caption{\protect\label{FigPasoRK}}
\end{figure}
