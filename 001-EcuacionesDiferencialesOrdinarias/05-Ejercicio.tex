\mode<article> 

Considere un péndulo sin rozamiento, oscilando en un plano. Calcule el período para ángulos
grandes. ¿Cuándo deja de valer la aproximación de ángulos pequeños? Tome las 
ecuaciones de la \autoref{FigurePendulo}

\begin{figure}[h]
  \includeslide[width=\textwidth]{FrameEcuacionesPendulo}
  \caption{\protect\label{FigurePendulo}   Péndulo con amplitudes arbitrarias   }
\end{figure}

\mode*

\begin{frame}<presentation>[label=FrameEcuacionesPendulo]
  \frametitle{Péndulo en ángulos grandes}
  \center
  \begin{minipage}{6cm}
      \begin{equation}
  \begin{aligned}
    m R \dfrac{d^2\theta}{dt} + mg sin(\theta)=0 \\
    v = \dfrac{d\theta}{dt}\\
    \dfrac{dv}{dt} = -\dfrac{g}{R} sin(\theta) \\
    \dfrac{d}{dt} 
    \begin{bmatrix} \theta\\  v \end{bmatrix}
    =
    \begin{bmatrix} v \\ - \dfrac{g}{m} sin(\theta) \end{bmatrix}
  \end{aligned}
\end{equation}

  \end{minipage}
  \onslide<1>
  \begin{minipage}{6cm}
      \includegraphics[width=\textwidth, trim=2cm 8cm 3.5cm 4cm, clip ]{pendulo.pdf}
  \end{minipage}
\end{frame}

