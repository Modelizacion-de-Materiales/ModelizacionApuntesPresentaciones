Vamos a implementar la solución de la ecuación diferencial de movimiento 
(a.k.a. segunda ley de Newton) para resolver un porblema muy conocido para 
nosotros. 

Tomemos un Paracaidista que se arroja en caída libre desde una altura inicial $y_0$.
El rozamiento del paracaídas con el aire le imrpime una fuerza de rozamiento proporcional 
a la velocidad $f = \gamma (v)$, de manera que la ecuación diferencial que gobierna su movimiento es

\begin{equation}\label{EqnNewtonParac}
  \dfrac{ d v }{dt} = g - \dfrac{\gamma (v)}{m} v
\end{equation}

donde $v$ y $m$ son la velocidad y la masa del paracaidista, $g$ es la aceleración de la gravedad.
Recordamos que debemos identificar la función característica de la ecuación diferencial, 

\begin{equation}
  \begin{aligned}
    \dfrac{d v}{d t} &= f(t, v) \\
    f(t, v) &= g - \gamma (v)
  \end{aligned}
\end{equation}

Si consideramos que el coeficiente de rozamiento depende de la velocidad como $\gamma (v) = k $
podemos obtener la solución exacta con facilidad.

\begin{equation}
  v = \frac{m g}{k} \Big[ \exp\Big( - \frac{k}{m} t \Big) -1 \Big]
\end{equation}

Tomemos la ecuación como de orden 1 en velocidad. 
Para hallar la solución de las velocidades en forma discreta, 
identificamos primero a la función característica a un tiempo $i$ cualquiera,

\begin{equation}\label{EqnFuncParac}
  f(t_i, v_i) = g-\frac{k}{m} v_{i-1}
\end{equation}

\noindent con lo cual podemos escribir los pasos para resolver con los métodos de 
Euler y de Runge Kutta. 

Si quisieramos tomar la ecuación como de orden dos en altura, debemos hacer el cambio
de variables

\begin{equation}\label{EqnParacAltura}
  \begin{aligned}
    h &= y
    v &= y',
    F(t, h, v) = \begin{pmatrix} v\\ f(t, v)    \end{pmatrix}
  \end{aligned}
\end{equation}

Por ejemplo entonces aplicando Euler,

\begin{equation}\label{EqnEulerOnHeight}
  \begin{pmatrix}  y \\   y '  \end{pmatrix}_i = 
    \begin{pmatrix}  y \\   y '  \end{pmatrix}_{i-1} +
      dt F \Big( t_{i-1} , h_{i-1}, v_{i-1} \Big)=
      \begin{pmatrix}  y \\   y '  \end{pmatrix}_{i-1} +
	dt \begin{pmatrix} v_{i-1} \\ f(t_{i-1} , v_{i-1} ) \end{pmatrix}
\end{equation}


