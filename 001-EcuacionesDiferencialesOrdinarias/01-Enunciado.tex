Vamos a implementar la solución de la ecuación diferencial de movimiento 
(a.k.a. segunda ley de Newton) para resolver un porblema muy conocido para 
nosotros. 

Tomemos un Paracaidista que se arroja en caída libre desde una altura inicial (supongamos 2000m).
El rozamiento del paracaídas con el aire le imrpime una fuerza de rozamiento proporcional 
a la velocidad, de manera que la ecuación diferencial que gobierna su movimiento es

\begin{equation}
  \dfrac{ d v }{dt} = g - \dfrac{\gamma}{m} v
  \label{EqnEDO1}
\end{equation}


