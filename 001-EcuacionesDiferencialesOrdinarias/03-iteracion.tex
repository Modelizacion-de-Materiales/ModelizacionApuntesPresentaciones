\mode<article>

Las funciones \texttt{pasoEU} y \texttt{pasoRK} resuelven un paso de propagación de los métodos
de Euler y de Runge Kutta de orden 4, respectivamente. El objetivo es \emph{iterar} para resolver
varios pasos de tiempo. Luego, podremos comparar las soluciones propagadas a la solución 
teórica

\mode*

\begin{frame}<presentation>[label=FrameIteracion]
  \frametitle{Iteraciones}

  \begin{tikzpicture}[every line/.style={line width=1pt}]
    \node (entrance) at (-8,0) {};
    \node[circle, radius=1pt, draw=black!50,  minimum width=0pt] 
    (itera) at ($(entrance)+(4,0)$) {};
    \node[rectangle, draw, line width=2pt,  minimum width=2cm, minimum height=1cm]
    (method) at (0,0) {pasoEU};
    \path[draw, ->] (entrance) -- (itera.west) node[midway, above] {$x_0, y_0, t_0, t_{max}$};
    \path[draw, ->] (itera.east) -- (method.west) node[midway, above] {$x_{i}, y_{i}, t_{i}$} ;
    \node[diamond, shape aspect=2, draw] (decide) at ($(method)+(3,0)$) {\tiny $t_i<t_{max}$};
    \path[draw, ->] (method.east) -- (decide.west) ;
    \path[draw, ->] (decide.south)
    -- ($(decide)+ (0, -3)$)  node[midway, left] {no}
    -- ($(itera) + (0, -3)$) node[midway, below] { $x_{i-1}, y_{i-1}, t_{i-1}$ }
    -- (itera.south);
  \end{tikzpicture}
\end{frame}
