\mode<article>

Las funciones \texttt{pasoEU} y \texttt{pasoRK} resuelven un paso de propagación de los métodos
de Euler y de Runge Kutta de orden 4, respectivamente. El objetivo es \emph{iterar} para resolver
varios pasos de tiempo. Luego, podremos comparar las soluciones propagadas a la solución 
teórica. 	

En la \autoref{FigSolucionAMano} vemos la diferencia entre los métodos para 
$\Delta t = 2s$. 
Debemos notar que la solución para Runge Kutta es prácticamente indistinguible frente a la solución teórica. 
Por otro lado, la solución por Euler se separa desde el primer paso y ese error se propaga en toda la 
solución.


\begin{figure}[h]

  \includeslide[width=\textwidth]{FrameSolucionAMano}
  \caption{\protect\label{FigSolucionAMano}
  Solución a mano para unas pocas iteraciones con los métodos vistos para un $\Delta t=2s$}
\end{figure}

Pensemos ahora cómo obtener una solución completa del movimiento del paracaidista. 
La iteración debe seguir dentro de los límites con sentido físico, por ejemplo 
hasta que el objeto halla llegado al piso (altura cero). 
Este resultado se aprecia en la \autoref{FiguraVelocidades} donde se aprecia también
que el móvil alcanza su velocidad límite

\begin{figure}[h]

  \includeslide[width=\textwidth]{FrameResolucionVelocidades}

  \caption{\protect\label{FiguraVelocidades} 
  Solución de las velocidades en todo el movimiento (hasta el piso)
  }

\end{figure}
\mode*

\begin{frame}<presentation>[label=FrameIteracion]
  \frametitle{Iteraciones}
\center
  \begin{tikzpicture}[every line/.style={line width=1pt}]
    \coordinate (Lx) at (3,0);
    \coordinate (Ly) at (-3,0);
    \coordinate (entrance) at (-9,0);
    \node[circle, radius=1pt, draw=black!50,  minimum width=0pt] 
    (itera) at ($(entrance)+(Lx)$) {};
    \node[rectangle, draw, line width=2pt,  minimum width=2cm, minimum height=1cm]
    (method) at ($(itera)+(Lx)$) {pasoEU};
    \path[draw, ->] (entrance) -- (itera.west) node[midway, above] {$x_0, y_0, t_0, t_{max}$};
    \path[draw, ->] (itera.east) -- (method.west) node[midway, above] {$x_{i}, y_{i}, t_{i}$} ;
    \node[diamond, shape aspect=2, draw] (decide) at ($(method)+(3,0)$) {\tiny $t_i<t_{max}$};
    \path[draw, ->] (method.east) -- (decide.west) ;
    \path[draw, ->] (decide.south)
    -- ($(decide)+ (0, -3)$)  node[midway, left] {no}
    -- ($(itera) + (0, -3)$) node[midway, below] { $x_{i-1}, y_{i-1}, t_{i-1}$ }
    -- (itera.south);
    \path[draw, ->] (decide.east) -- ($(decide.east)+(1,0)$) 
    node[midway, above] {si} 
    node[anchor=west, draw] {\texttt{termina}};
  \end{tikzpicture}
\end{frame}

\begin{frame}<presentation>[label=FrameSolucionAMano]
  \frametitle{Iteramos a mano unas pocos pasos}
  \begin{columns}
    \column{0.5\textwidth}
    \begin{codeblock}
	\VerbatimInput{CODES/Table_A_Mano.dat}
    \end{codeblock}
    \column{0.5\textwidth}
    \includegraphics[width=\textwidth]{RESULTS/comparacion.pdf}
  \end{columns}

\end{frame}

\begin{frame}<presentation>[label=FrameResolucionVelocidades]
  \frametitle{Velocidades en todo el rango}
\center
  \includegraphics[height=5cm, trim=0cm 9cm 0cm 8cm]{RESULTS/velocidades-paso_200.pdf}

\end{frame}
\mode<all>
