\mode<article>

Lo que se hizo hastá aquí no es ni mas ni menos que plantear un sistema lineal
de ecuaciones de la forma

\mode*
\begin{frame}<1>[label=FrameNotacionGeneral]
  \frametitle<presentation>{Generalización del Problema}

  \mode<presentation>{Sistema de ecuaciones:}
  \begin{equation}
    K_{i,j} u_j  = f_i
  \end{equation}

\onslide<2->

  \mode<presentation>{
  \small consideramos las las ecuaciones correspondientes a los desplazamientos ingógnita
}
  \begin{equation}
    K_{r,j} u_j  = f_r
  \end{equation}

  \onslide<3->

  \mode<presentation>{ \small Donde $r$  es un índice que recorre dichos grados de libertad
  (incógnitas). Como j recorre todos los grados de libertad, podemos separar aquellos vinculados
  de los libres (incógnitas). Tomemos $r'$ recorriendo las incógnitas, $s$ recorriendo los
  vínculos, }

  \begin{equation}
    K_{r, r'} u_{r'}  + K_{r,s} u_s = F_r
  \end{equation}

\end{frame}

\mode<article>

En este conjunto de cuaciones existe un subconjunto que corresponden a
aquellas donde los desplazamientos  $u$  son incógnitas y las fuerzas $F$  son dadas.
Nombremos $r$ al índice que define este subconjunto, de manera que las ecuaciones
pueden reescribirse Como

\mode*

\againframe<2>{FrameNotacionGeneral}

\mode<article>

 Como el índice j recorre todos los grados de libertad, es posible separar o particionar este
 índice entre los grados de libertad vinculados $s$ y las incógnitas $r'$, 
 de manera que el sistema puede reorderarse de
manera de separar un término para los vínculos y otro para las incógnitas, 

\mode*

\againframe<3>{FrameNotacionGeneral}

 Por último, puede reordenarse de manera que solo
aparezcan los grados de libertad incógnita a la derecha de la ecuación,

\begin{frame}[label=FrameSolveGeneral]
  \frametitle<presentation>{Solución General}

  \mode<presentation>{ Por último }
  \begin{equation}\label{EqReordenoGeneral}
    K_{r,r'} u_{r'} = F_r - K_{r,s} u_s
  \end{equation}

  que se resuelve rápidamente:

  \begin{equation}
    u_{r'} = K_{r',r} \left[ F_r - K_{r,s} u_s \right]
  \end{equation}

  Una vez que se conocen todos los $u_r$a se pueden recuerar las fuerzas de vínculo

  \begin{equation}
    F_s = K_{s,j} u_j
  \end{equation}

\end{frame}


\mode<all>
