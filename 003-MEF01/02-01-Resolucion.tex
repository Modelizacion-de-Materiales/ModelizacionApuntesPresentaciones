\mode<article>

Recordemos las condiciones de contorno. El nodo 1 se encuentra empotrado por lo
que su desplazamiento está fijo en $x_1=0$. El nodo 4 está obligado a
desplazarse $x_4 = \delta$ de su posición de equilibrio.  La fuerza necesaria
para que estos dos nodos permanezcan en esas condiciones, $F_1$ y $F_4$ son
desconocidas. 

Por otro lado, los nodos 2 y 3 se han desplazado una cantidad desconocida, pero
como suponemos que están en equilibrio estático podemos decir que no hay
fuerzas externas aplicadas en ellos, $F_2 = F_3 = 0$.

En principio, el sistema lineal encontrado representa un sistema de ecuaciones.
Tratemos primero las ecuaciones que corresponden con los desplazamientos
desconocidos $x_2$ y $x_3$ en el ejemplo del resorte. La operatoria consiste en
expandir las ecuaciones correspondientes en todos los grados de libertad, tal y
como aparecen en la ecuación \ref{EqSistemaLinealExpandido}.

Luego deben ser reordenados los términos de los grados de libertad vinculados
$x_1$ y $x_4$ a la derecha del igual, de manera de dejar del lado izquierdo del
igual solo a los términos de los grados de libertad incógnita $x_2$ y $x_3$, 

\mode*

\begin{frame}<presentation>[label=FrameSeparacionSistema]
  \frametitle{Separación del sistema lineal}
  \begin{tikzpicture}
    [      overlay,remember picture
%      inc/.style={fill=red, opacity=0.2},
%      vin/.style={fill=blue, opacity=0.2}
    ]
    \coordinate (O) at (1,0);
    \coordinate (E) at (10,-4);
    \coordinate (L1) at (1,0.5);
    \coordinate (L2) at (3,0.5);
    \coordinate (A) at (1.5,-1 );
    \coordinate (B) at (2.5,-1);
    \coordinate (C) at (5.5, -1);
    % las coordenadas de xvin y xinc
    \coordinate (X4) at (7,-2);
    \coordinate (X1) at (7,-0.5);
    \coordinate (X3) at ($(X4)+(0,0.5)$);
    \coordinate (F3) at (8.5,-1.5);
    \coordinate (F2) at ($(F3)+(0,0.5)$);
    \coordinate (LI) at (1,1);
   % esta es el desplazamiento para el renglon de abajo 
    \coordinate (D) at (0,-0.5); 
    \onslide<2->
    \draw[vin] (X4) rectangle ($(X4)+(L1)$);
    \draw[vin] (X1) rectangle ($(X1)+(L1)$);
    \draw[inc] (X3) rectangle ($(X3)+(LI)$);
    \onslide<2>
    \draw[vin] (A)  rectangle ($(A)+(L1)$);
    \draw[inc] (B)  rectangle ($(B)+(L2)$); 
    \draw[vin] (C)  rectangle ($(C)+(L1)$);
    \draw[vin] (F2) rectangle ($(F2)+(L1)$);
    \onslide<3>
    \draw[vin] ($(D)+(A)$) rectangle ($(A)+(L1)+(D)$);
    \draw[inc] ($(D)+(B)$) rectangle ($(B)+(L2)+(D)$); 
    \draw[vin] ($(D)+(C)$) rectangle ($(C)+(L1)+(D)$);
    \draw[vin] (F3) rectangle ($(F3)+(L1)$);
    \node at (0,0) {};

  \end{tikzpicture} %dejar esta parte junta para que coincidan las posiciones.
  \begin{equation} \label{eqSistemaLineal2}
    \matriz \vectorx  =  \vectorF
  \end{equation}



\onslide<2->{
\begin{equation}\begin{aligned}
    \ecuaciona \\ 
  \onslide<3->{
    \ecuacionb
  }
\end{aligned}\end{equation}
}
\end{frame}

\begin{frame}[label=FrameReordenarEcuaciones23]
  \frametitle<presentation>{Reordenando Ecuaciones Para Incógnitas}

\begin{equation}  \label{EqReordenoIncognitas}
    \begin{aligned}
      \ecuaciona \\  \ecuacionb \\
      \quad \Rightarrow \quad & \\
      \ecuacionao \\ \ecuacionbo
    \end{aligned}
\end{equation}
\onslide<2>
  \mode<presentation>{
  \begin{tikzpicture}[overlay, remember picture]
    %\draw (0,0) grid (11,4);
    \draw[inc] (2.5,0.5) rectangle (6.6,1.5);
    \draw[vin] (7,0.5) rectangle (11,1.5);
  \end{tikzpicture}
}

\end{frame}

\mode<all>

\subsection{Solución de las incógnitas}

\mode<article>
Finalmente, pensemos en un vector de incógnitas $X_r = (x_2, x_3)$   y un
vector de vínculos $X_s  = (x_1, x_4)$ . Con estas definiciones, las
ecuaciones (\ref{EqReordenoIncognitas})  puede escribirse en forma matricial, 

\mode*

  \begin{frame}[label=FrameSistemaReducido]
    \frametitle<presentation>{Reducción del Sistema}
    \begin{equation}\label{EqSistemaReducido}
      \underbrace{
	\begin{pmatrix} k_1 + k_2 & - k_2 \\ -k_2 & k_1 + k_2 \end{pmatrix}
	}_{ \mathbf{K_{red} } }
	\begin{pmatrix}x_2 \\ x_3\end{pmatrix}
	  = 
	  \begin{pmatrix}F_2 \\ F_3\end{pmatrix}
	    -
	  \underbrace{
	      \begin{pmatrix} -k_1 & 0 \\ 0 & -k1 \end{pmatrix}
	   }_{ \mathbf{ K_{vin} } }
	\begin{pmatrix}x_1 \\ x_4\end{pmatrix}
    \end{equation}

    \onslide<2>
    \mode<presentation>{    Resolvemos:}
    \begin{equation} \label{EqSolucionSistema}
      \begin{pmatrix} x_2 \\ x_3  \end{pmatrix}
	=
	\mathbf{K_{red}} ^{-1} 
	\left[
      \begin{pmatrix} F_2 \\ F_3  \end{pmatrix}
	- 
	\mathbf{ K_{vin} }
      \begin{pmatrix} x_1 \\ x_4  \end{pmatrix}
	\right]
    \end{equation}

  \end{frame}

  \mode<article>

de manera que quedarán definidas la \textbf{matriz de rigidez reducida} $\mathbf{K_{red} }$ y la
\textbf{matriz de coeficientes vínculos} $\mathbf{K_{vin}}$. La solución del sistema puede
hallarse en forma inmediata como se muestra en la ecuación (\ref{EqSolucionSistema}).

\mode*

\mode<all>
