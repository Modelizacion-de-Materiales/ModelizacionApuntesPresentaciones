\mode<article>

Recordemos las condiciones de contorno. El nodo 1 se encuentra empotrado por lo
que su desplazamiento está fijo en $x_1=0$. El nodo 4 está obligado a
desplazarse $x_4 = \delta$ de su posición de equilibrio.  La fuerza necesaria
para que estos dos nodos permanezcan en esas condiciones, $F_1$ y $F_4$ son
desconocidas. 

Por otro lado, los nodos 2 y 3 se han desplazado una cantidad desconocida, pero
como suponemos que están en equilibrio estático podemos decir que no hay
fuerzas externas aplicadas en ellos, $F_2 = F_3 = 0$.

En principio, el sistema lineal encontrado representa un sistema de ecuaciones.
Tratemos primero las ecuaciones que corresponden con los desplazamientos
desconocidos $x_2$ y $x_3$ en el ejemplo del resorte. La operatoria consiste en
expandir las ecuaciones correspondientes en todos los grados de libertad, tal y
como aparecen en la ecuación \ref{EqSistemaLinealExpandido}.

Luego deben ser reordenados los términos de los grados de libertad vinculados
$x_1$ y $x_4$ a la derecha del igual, de manera de dejar del lado izquierdo del
igual solo a los términos de los grados de libertad incógnita $x_2$ y $x_3$, 

\begin{equation}
  \label{EqReordenoIncognitas}
  %\begin{split}
    \begin{aligned}
        F_2 &= x_1 (-k_1) + x_2(k_1+k_2)+ x_3(-k_2) + x_4(0)\\
       F_3 &= x_1 (0) + x_2(0)+ x_3(-k_3) + x_4(k_3) \\
      \quad \Rightarrow \quad & \\
	 x_2(k_1+k_2)+ x_3(-k_2) &= F_2  - (x_1 (-k_1) +  x_4(0)   )\\
         x_2(0)      + x_3(-k_3) &= F_3  - (x_1 (0)    +  x_4(k_3) )
    \end{aligned}
\end{equation}

Finalmente, pensemos en un vector de incógnitas $X_r = (x_2, x_3)$   y un
vector de vínculos $X_s  = (x_1, x_4)$ . Con estas definiciones, las
ecuaciones (\ref{EqReordenoIncognitas})  puede escribirse en forma matricial, de manera que
  quedarán definidas la \textbf{matriz de rigidez reducida} $\mathbf{K_{red} }$ 
  y la \textbf{matriz de coeficientes vínculos} $\mathbf{K_{vin}}$

\begin{equation}
  \underbrace{
  \begin{pmatrix} k_1 + k_2 & - k_2 \\ -k_2 & k_1 + k_2 \end{pmatrix}
  }_{ \mathbf{K_{red} } }
  \begin{pmatrix}x_2 \\ x_3\end{pmatrix}
  = 
  \begin{pmatrix}F_2 \\ F_3\end{pmatrix}
  -
  \underbrace{
  \begin{pmatrix} -k_1 & 0 \\ 0 & -k1 \end{pmatrix}
  }_{ \mathbf{ K_{vin} } }
  \begin{pmatrix}x_1 \\ x_4\end{pmatrix}
\end{equation}


\mode*

\mode<all>
