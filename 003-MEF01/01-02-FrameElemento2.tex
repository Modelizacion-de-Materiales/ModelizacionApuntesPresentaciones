\begin{frame}[label=FrameFuerzasElemento2]
  \frametitle<presentation>{Fuerzas sobre el Elemento 2}

  \begin{figure}
    \includegraphics[width=\textwidth,page=4, trim=5cm 8cm 5cm 6cm, clip=true]
    {./Libreoffice/MEF01_2018.pdf}
    \mode<article>{
      \caption{\protect\label{FigureFuerzasElemento2} Fuerzas sobre el elemento 2}
    }
  \end{figure}

  \begin{equation} 
    \label{EqElemento2}
    \begin{split}
      f_1^2 &= -k_2 (x_3 - x_2)\\[10pt]
      f_2^2 &= k_2 (x_3 - x_2)
    \end{split}
    \quad \Rightarrow \quad
     \begin{pmatrix}
       f_1^2\\[10pt]
       f_2^2
     \end{pmatrix}
     =
     \underbrace{
       k_2 
       \begin{pmatrix}
	 1 & -1 \\[10pt]
	 -1 & 1 
       \end{pmatrix}
     }_{ \mathbf{ k_2 ^{el} } }
    \begin{pmatrix}
      x_2 \\[10pt]
      x_3
    \end{pmatrix}
%    
  \end{equation}
          
\end{frame}

