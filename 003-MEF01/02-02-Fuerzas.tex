\mode<article>

El problema no está terminado aún puesto que falta resolver las fuerzas $F_1$ y
$F_4$. Pero, conociendo los desplazamientos a partir de la ecuación
\ref{EqSolucionSistema}, resulta trivial resolver a partír de las ecuaciones
correspondientes en el sistema de ecuaciones \ref{EqSistemaReducido}. Aún así,
por completitud, escribamos la forma matricial de las ecuaciones
correspondientes. Tomemos un vector de fuerzas de vínculo $F_s = (F_1, F_4)$,
el sistema lineal equivalente al subconjunto de ecuaciones correspondientes es 

\mode*
\begin{frame}[label=FrameSolveForces]
  \frametitle<presentation>{Solución de las fuerzas dev vínculo}
  \mode<presentation>{
  \begin{equation}    \begin{aligned}
      \matriz \vectorx = \vectorF
    \end{aligned}  \end{equation}
  \begin{tikzpicture}[overlay, remember picture]
    \draw (0,0) grid (10,2);
    \coordinate (LX) at (5,0.5);
    \coordinate (A)  at (1.5,1.7);
    \coordinate (B)  at (1.5,0.3);
    \coordinate (C)  at (6.8, 0) ;
    \coordinate (LY) at (1,2.3) ;
    \coordinate (F4) at (8.5, 0.3);
    \coordinate (F1) at (8.5, 1.8);
    \coordinate (LF) at(0.5, 0.5);
    \draw[vin]  (A) rectangle ($(A)+(LX)$);
    \draw[vin]  (B) rectangle ($(B)+(LX)$);
    \draw[vin]  (C) rectangle ($(C) + (LY)$);
    \draw[vin]  (F1) rectangle ($(F1)+(LF)$);
    \draw[vin]  (F4) rectangle ($(F4)+(LF)$);
  \end{tikzpicture}
    \onslide<2->
  \begin{equation}
    \begin{aligned}
      \solvefa \\
      \solvefb
    \end{aligned}
  \end{equation}
}
  \onslide<3>

    \begin{equation}\label{EqSolucionFuerzasVinc}
      \begin{pmatrix}F_1 \\ F_4 \end{pmatrix} 
	= 
	\begin{pmatrix} k_1  & -k_1 & 0 & 0 \\ 0 & 0 & -k_3 & k_3 \end{pmatrix} 
      \vectorx
    \end{equation}


\end{frame}

\mode<all>
