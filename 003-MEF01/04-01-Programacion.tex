\mode<article>

La ventaja del planteo desarrollado arriba radica en que se
puede aplicar fácilmente a cualquier lenguaje de programación moderno,i.e.
fortan, python, matlab, o el que se le ocurra.  Es posible entender que cada
una de las listas de grados de libertad definidas pueden escribirse como
vectores. Cada componente del vector es el índice de un grado de libertad del
problema que corresponda.


\mode*

\begin{frame}[label=FrameIndicesProblema]
  \frametitle<presentation>{Problema del Resorte}

  \mode<Presentación>{
   \includegraphics[width=\textwidth,page=2, trim=5cm 8cm 5cm 6cm, clip=true]
   {./Libreoffice/MEF01_2018.pdf}
  }

  \begin{equation}
    \begin{split}
    i, j = ( 1, 2, 3, 4) \\
    r = ( 2, 3) ; s = (1 , 4) 
    \end{split}
  \end{equation}

\end{frame}

\mode<all>

\mode<article> 

De esta forma es posible implementar la ecuación general de la ecuación. 
Para eso debemos hacer uso de los llamados \emph{slice} de un vector o array. 
En \emph{python} deberemos hacer uso de la función \texttt{numpy.ix\_} que nos permitirá
tomar un \emph{slice} bidimensional de una matriz dados los vectores de índices \texttt{r}
y \texttt{ s}, como se ilustra en la Figura \ref{FiguraProgramarMatrices}

\begin{figure}
  \includeslide[width=\textwidth]{FrameMatricesProgramadas}
  \mode<article>{
  \caption{\protect\label{FiguraProgramarMatrices} Códgio para representar la solución del sistema mixto 
  por reducción de matrices}
}


\end{figure}

\mode<all>

\mode*

\begin{frame}<presentation>[label=FrameMatricesProgramadas]
  \frametitle{Programacion de la solución}

%    \begin{codeblock}
  \ttfamily
  \lstinputlisting[language=Python, style=codeblock]{codesolve.txt}
      %\verbatiminput{codesolve.txt}
%    \end{codeblock}

\end{frame}

\mode<all>
