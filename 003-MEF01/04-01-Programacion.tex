\mode<article>

La ventaja del planteo desarrollado arriba radica en que se
puede aplicar fácilmente a cualquier lenguaje de programación moderno,i.e.
fortan, python, matlab, o el que se le ocurra.  Es posible entender que cada
una de las listas de grados de libertad definidas pueden escribirse como
vectores. Cada componente del vector es el índice de un grado de libertad del
problema que corresponda. En nuestro ejemplo del resorte

\mode*

\begin{frame}[label=FrameIndicesProblema]
  \frametitle<presentation>{Problema del Resorte}

  \mode<Presentación>{
   \includegraphics[width=\textwidth,page=2, trim=5cm 8cm 5cm 6cm, clip=true]
   {./Libreoffice/MEF01_2018.pdf}
  }

  \begin{equation}
    \begin{split}
    i, j = ( 1, 2, 3, 4) \\
    r = ( 2, 3) ; s = (1 , 4) 
    \end{split}
  \end{equation}

\end{frame}

Si se ha definido la matriz de rigidez del problema, puede definirse la
\emph{matriz de rigidez reducida}
del sistema de la siguiente forma:

\begin{frame}<presentation>[label=FrameMatricesProgramadas]
  \frametitle{}

\end{frame}

y la matriz vinculada,

De esta forma es posible implementar la ecuación general de la ecuación

en forma inmediata,

Esto resuelve los desplazamientos desconocidos. Para resolver las
fuerzas de vínculo, directamente se aplica:


\mode*

\mode<all>
