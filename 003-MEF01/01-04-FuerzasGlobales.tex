\mode<article>

La fuerza total sobre cada uno de los nodos es la suma de todas las fuerzas que actúa sobre cada
uno de ellos. Al escribir estas ecuaciones globales, es posible acoplar a todos los grados de
libertad a partir de las ecuaciones locales, expandiendo cada una de ellas de manera que
aparezcan en forma explícita todos los grados de libertad del problema $x_1$, $x_2$, $x_3$ y
$x_4$. La fuerza global sobre cada nodo puede numerarse $F_1$, $F_2$, $F_3$ y $F_4$ .

\mode*

\begin{frame}[label=FrameFuerzasGlobales]
  \frametitle<presentation>{Fuerzas Globales}

  \begin{figure}
    \fbox{
    \includegraphics[width=\textwidth,page=6, trim=5cm 11cm 3cm 7cm, clip=true]
    {./Libreoffice/MEF01_2018.pdf}
  }
    \mode<article>{
      \caption{\protect\label{FigureFuerzasGlobales} Fuerzas Globales}
    }
  \end{figure}
%
 \tiny 
  \begin{equation}
    \begin{aligned}
      F_1 &= f_1^1  = -k_1( x_2-x_1) =
                   x_1 (k_1) + x_2(-k_1)+ x_3(0) + x_4(0)\\
      F_2 &= f_1^2 + f_2^1 = k_1( x_2-x_1) -k_2(x_3 - x_2) = 
                   x_1 (-k_1) + x_2(k_1+k_2)+ x_3(-k_2) + x_4(0)\\
      F_3 &= f_2^2 + f_1^3 = k_2( x_3-x_2) - k_3 (x_4 - x_3) = 
                    x_1 (0) + x_2(-k_2)+ x_3(k_2 + k_3) + x_4(-k_3)\\
      F_4 &= f_2^3 = k_3( x_4-x_3)=
                     x_1 (0) + x_2(0)+ x_3(-k_3) + x_4(k_3)
    \end{aligned}
  \end{equation}
%
\end{frame}

\mode<article>

Este sistema de ecuaciones puede pensarse en forma Lineal, donde tenemos un
vector de Fuerzas $\mathbf{F}$  y un vector de desplazamientos $\mathbf{x}$. Notar que también
queda definida la matriz de rigidez global delsistema $\mathbf{M}$

\mode*

\begin{frame}[label=FrameMatrizRigidez]
  \frametitle<presentation>{Matriz de Rigidez Global}

  \begin{equation}
    \underbrace{
      \begin{pmatrix}
	k_1  & -k_1     & 0    & 0 \\
	-k_1 & k_1 +k_2 & -k_2 & 0 \\
	0    & -k_2     & k_2 + k_3 & -k3\\
	0    &  0       & -k_3      & k_3
      \end{pmatrix}
    }_{\mathbf{M}}
    \begin{pmatrix}
      x_1\\
      x_2\\
      x_3\\
      x_4
    \end{pmatrix}
    =
    \begin{pmatrix}
      F_1\\
      F_2\\
      F_3\\
      F_4
    \end{pmatrix}
  \end{equation}

\end{frame}

\mode<all>
