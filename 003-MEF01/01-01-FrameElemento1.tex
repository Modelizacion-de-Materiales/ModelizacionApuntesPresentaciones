\mode*

\begin{frame}[label=FrameFuerzasElemento1]
  \frametitle<presentation>{Fuerzas sobre el Elemento 1}

  \begin{figure}
    \includegraphics[width=\textwidth,page=3, trim=5cm 8cm 5cm 6cm, clip=true]
    {./Libreoffice/MEF01_2018.pdf}
    \mode<article>{
      \caption{\protect\label{FigureFuerzasElemento1} Fuerzas sobre el elemento 1}
    }
  \end{figure}

  \begin{equation} 
    \label{EqElemento1}
    \begin{split}
      f_1^1 &= -k_1 (x_2 - x_1)\\[10pt]
      f_2^1 &= k_1 (x_2 - x_1)
    \end{split}
    \quad \Rightarrow \quad
     \begin{pmatrix}
       f_1^1\\[10pt]
       f_2^1
     \end{pmatrix}
     =
     \underbrace{
       k_1 
       \begin{pmatrix}
	 1 & -1 \\[10pt]
	 -1 & 1 
       \end{pmatrix}
     }_{ \mathbf{ k_1 ^{el} } }
    \begin{pmatrix}
      x_1 \\[10pt]
      x_2
    \end{pmatrix}
%    
  \end{equation}
          
\end{frame}

\mode<all>
