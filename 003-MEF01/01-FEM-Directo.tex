\mode<article>

Consideremos el problema 1 de la guía, en el que una cadena de tres resortes, empotrada en el
extremo izquierdo, es solicitada por un desplazamiento en el extremo derecho.

Las condiciones sobre los extremos nombradas anteriormente nos dan las condiciones de contorno
sobre la cadena de resortes.

Para poder resolver la fuerza necesaria para efectuar el desplazamiento usando el método de
elementos finitos, dividimos el problema de forma natural en tres elementos. Cada resorte será un
elemento, y cada elemento tendrá dos nodos, de modo que todos los grados de libertad del problema
pueden representarse en cuatro nodos. Cada nodo marca un extremo de un resorte. 

Los numeremos los Elementos de 1 a 3 indicados con cajas en la figura, y los nodos de 1 a 4.
indicados con círculos en la Figura \ref{FiguraNumeracionGL}


\mode*
\begin{frame}[label=FrameNumeracionGL]
  \frametitle<presentation>{Problema de los Resortes}

  \begin{figure}
    \includegraphics[width=\textwidth,page=2, trim=5cm 7cm 5cm 8cm, clip=true]
    {./Libreoffice/MEF01_2018.pdf}
    \mode<article>{
      \caption { Esquema del problema \protect\label{FiguraNumeracionGL} }
    }
  \end{figure}


\end{frame}

\mode<article>

\subsection{Fuerzas en los Nodos}

Consideremos las fuerzas que cada Elemento o resorte ejerce sobre los nodos, tal cual se hizo en
la clase teórica.

\mode<all>
