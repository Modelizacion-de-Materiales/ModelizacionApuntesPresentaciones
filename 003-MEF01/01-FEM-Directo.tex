\mode<article>

Consideremos el problema 1 de la guía, en el que una cadena de tres resortes, empotrada en el
extremo izquierdo, es solicitada por un desplazamiento en el extremo derecho.

Las condiciones sobre los extremos nombradas anteriormente nos dan las condiciones de contorno
sobre la cadena de resortes.

Para poder resolver la fuerza necesaria para efectuar el desplazamiento usando el método de
elementos finitos, dividimos el problema de forma natural en tres elementos. Cada resorte será un
elemento, y cada elemento tendrá dos nodos, de modo que todos los grados de libertad del problema
pueden representarse en cuatro nodos. Cada nodo marca un extremo de un resorte. 

Los numeremos los Elementos de 1 a 3 indicados con cajas en la figura, y los nodos de 1 a 4.
indicados con círculos en la Figura \ref{FiguraNumeracionGL}

\mode*

\begin{frame}[label=FrameNumeracionGL]
  \frametitle<presentation>{Problema de los Resortes}

  \begin{figure}
    \includegraphics[width=\textwidth,page=2, trim=5cm 7cm 5cm 8cm, clip=true]
    {./Libreoffice/MEF01_2018.pdf}
    \mode<article>{
      \caption { Esquema del problema \protect\label{FiguraNumeracionGL} }
    }
  \end{figure}


\end{frame}

\mode<all>

\mode<article>


Consideremos las fuerzas que cada Elemento o resorte ejerce sobre los nodos, tal cual se hizo en
la clase teórica. Agreguemos ahora un poco de detalle sobre cada uno de los resortes.

%\subsubsection{Fuerzas sobre el elemento 1}
%
%Como se ilustra en la Figura \ref{FigureFuerzasElemento1} Las fuerza sobre el nodo 1 del Elemento
%1 es $f_1^1$ , mientras que la fuerza sobre el nodo 2 del Elemento 1 es $f_2^1$. Ambas fuerzas
%son de igual módulo pero opuestas, y quedan determinadas por el estiramiento del resorte. Este
%estiramiento será directamente la diferencia de corrimientos entre los nodos que determinan el
%elemento 1 y 2. Si las posiciones de dichos nodos son $x_1$ y $x_2$, se pueden escribir las
%fuerzas como en la ecuación (\ref{EqElemento1}). 
%
%\mode*
%
%\begin{frame}[label=FrameFuerzasElemento1]
%  \frametitle<presentation>{Fuerzas sobre el Elemento 1}
%
%  \begin{figure}
%    \includegraphics[width=\textwidth,page=3, trim=5cm 8cm 5cm 6cm, clip=true]
%    {./Libreoffice/MEF01_2018.pdf}
%    \mode<article>{
%      \caption{\protect\label{FigureFuerzasElemento1} Fuerzas sobre el elemento 1}
%    }
%  \end{figure}
%
%  \begin{equation} 
%    \label{EqElemento1}
%    \begin{split}
%      f_1^1 &= -k_1 (x_2 - x_1)\\[10pt]
%      f_2^1 &= k_1 (x_2 - x_1)
%    \end{split}
%    \quad \Rightarrow \quad
%     \begin{pmatrix}
%       f_1^1\\[10pt]
%       f_2^1
%     \end{pmatrix}
%     =
%     \underbrace{
%       k_1 
%       \begin{pmatrix}
%	 1 & -1 \\[10pt]
%	 -1 & 1 
%       \end{pmatrix}
%     }_{ \mathbf{ k_1 ^{el} } }
%    \begin{pmatrix}
%      x_1 \\[10pt]
%      x_2
%    \end{pmatrix}
%%    
%  \end{equation}
%          
%\end{frame}



% Notar que ha quedado definida la matriz de rigidez elemental del 
% Elemento 1, $\mathbf{k_1 ^{el} }$.

%\subsubsection{Elemento 2}
%
%Habiendo introducido la notación para
%el elemento 1, simplemente tenemos aquí las ecuaciones equivalentes para el
%Elemento 2. la diferencia será cuáles son los grados de libertad
%involucrados, porque el Elemento 2 ‘conecta’ o ‘acopla’ a los nodos 2 y 3, y por
%lo tanto son las posiciones de dichos nodos, $x_2$ y $x_3$ , las que determinarán el
%estiramiento del resorte.
%
%\mode*
%
%%\begin{frame}[label=FrameFuerzasElemento2]
  \frametitle<presentation>{Fuerzas sobre el Elemento 2}

  \begin{figure}
    \includegraphics[width=\textwidth,page=4, trim=5cm 8cm 5cm 6cm, clip=true]
    {./Libreoffice/MEF01_2018.pdf}
    \mode<article>{
      \caption{\protect\label{FigureFuerzasElemento2} Fuerzas sobre el elemento 2}
    }
  \end{figure}

  \begin{equation} 
    \label{EqElemento2}
    \begin{split}
      f_1^2 &= -k_2 (x_3 - x_2)\\[10pt]
      f_2^2 &= k_2 (x_3 - x_2)
    \end{split}
    \quad \Rightarrow \quad
     \begin{pmatrix}
       f_1^2\\[10pt]
       f_2^2
     \end{pmatrix}
     =
     \underbrace{
       k_2 
       \begin{pmatrix}
	 1 & -1 \\[10pt]
	 -1 & 1 
       \end{pmatrix}
     }_{ \mathbf{ k_2 ^{el} } }
    \begin{pmatrix}
      x_2 \\[10pt]
      x_3
    \end{pmatrix}
%    
  \end{equation}
          
\end{frame}


%\begin{frame}[label=FrameFuerzasElemento2]
%  \frametitle<presentation>{Fuerzas sobre el Elemento 2}
%
%  \begin{figure}
%    \includegraphics[width=\textwidth,page=4, trim=5cm 8cm 5cm 6cm, clip=true]
%    {./Libreoffice/MEF01_2018.pdf}
%    \mode<article>{
%      \caption{\protect\label{FigureFuerzasElemento2} Fuerzas sobre el elemento 2}
%    }
%  \end{figure}
%
%  \begin{equation} 
%    \label{EqElemento2}
%    \begin{split}
%      f_1^2 &= -k_2 (x_3 - x_2)\\[10pt]
%      f_2^2 &= k_2 (x_3 - x_2)
%    \end{split}
%    \quad \Rightarrow \quad
%     \begin{pmatrix}
%       f_1^2\\[10pt]
%       f_2^2
%     \end{pmatrix}
%     =
%     \underbrace{
%       k_2 
%       \begin{pmatrix}
%	 1 & -1 \\[10pt]
%	 -1 & 1 
%       \end{pmatrix}
%     }_{ \mathbf{ k_2 ^{el} } }
%    \begin{pmatrix}
%      x_2 \\[10pt]
%      x_3
%    \end{pmatrix}
%%    
%  \end{equation}
%\end{frame}


\mode<all>
