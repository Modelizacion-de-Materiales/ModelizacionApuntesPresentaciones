% modo presentación. si se cambia a modo notas se imprimen las notas (creo)
%\documentclass[xcolor={dvipsnames,x11names,svgnames},aspectratio=169,notes]{beamer}
% \mode<presentation>
%handout de las diapositivas, comentar arriva!
% handout de la presentacion
%\documentclass[handout,xcolor={dvipsnames,x11names,svgnames},aspectratio=169,notes]{beamer}
%\mode<handout>

% paquete beamer, opciones para nombres de colores y aspect ratio de panatalla grande.
%\documentclass[a4paper,12pt]{article}
%\usepackage{beamerarticle}

% para no tener problemas con acentos etc.
\usepackage[utf8]{inputenc}
% en español
\usepackage[spanish]{babel}
%matemática
\usepackage{amsmath}
% este no se si hace falta pero por las dudas
\usepackage{graphicx}
% para incluir peliculas
\usepackage{multimedia}
% para usar segunda pantalla
\usepackage{pgfpages}
\usepackage{pgf}
% para hacer dibujitos
\usepackage{tikz}
\usetikzlibrary[automata,calc,arrows,decorations.pathmorphing,backgrounds,shapes,
patterns,positioning,fit,petri,overlay-beamer-styles]
\tikzstyle{every picture}+=[remember picture]
%recuadros sencillos
%\usepackage{tcolorbox}
% enumeradores intercambiables
\usepackage{enumerate}
% para subtitulos en figuras
\usepackage{subcaption}
%listings for code input
\usepackage{listings}
%verbatim input from file!
\usepackage{verbatim}
\usepackage{fancyvrb}
% para modificar los encabezados y pies de página.
%\usepackage{fancyhdr}
%\pagestyle{fancy}
\usepackage{standalone}

\definecolor{codegreen}{rgb}{0,0.6,0}
\definecolor{codegray}{rgb}{0.5,0.5,0.5}
\definecolor{codepurple}{rgb}{0.58,0,0.82}
\definecolor{backcolour}{rgb}{0.95,0.95,0.92}

\lstdefinestyle{codeblock}{
    backgroundcolor=\color{backcolour},   
    commentstyle=\color{codegreen},
    keywordstyle=\color{magenta},
    numberstyle=\tiny\color{codegray},
    stringstyle=\color{codepurple},
    basicstyle=\ttfamily\footnotesize,
    breakatwhitespace=false,         
    breaklines=true,                 
    captionpos=b,                    
    keepspaces=true,                 
    numbers=left,                    
    numbersep=5pt,                  
    showspaces=false,                
    showstringspaces=false,
    showtabs=false,                  
    tabsize=2
}


% para mostrar las notas en modo presentacion. 
%\setbeameroption{show notes}
%para ocultar las notas
%\setbeameroption{hide notes}
%para dejar las notas en la segunda patnalla
%\setbeameroption{show notes on second screen}
%incluyo los paquetes necesarios

% en caso de handouts, ver 2 en 1 o 4 en 


% en forma arbitraria decid que los parrafos no llevan indentación.
\setlength{\parindent}{0cm}

% incluyo los beamercolors
\include{./PREAMBLE/BEAMERCOLORS}

%incluyo el tema y modificaciones
\include{./PREAMBLE/BEAMERTHEME}

% modifico los temas
\include{./PREAMBLE/THEMES}

%%%%%%%%%%%%%%%%%%%%%%%%%%%% newcommands
\newcommand{\ecuaciona}{  x_1 (-k_1) +  x_2(k_1+k_2)+  x_3(-k_2) +  x_4(0) &= F_2   }
\newcommand{\ecuacionao}{  x_2(k_1+k_2)+ x_3(-k_2)   &= F_2  -  (x_1 (-k_1) +   x_4(0)   )}
\newcommand{\ecuacionb}{ x_1 (0) +  x_2(0)+  x_3(-k_3) + x_4(k_3)    &= F_3   }
\newcommand{\ecuacionbo}{  x_2(0)  +  x_3(-k_3)  &= F_3  - (x_1 (0)    +  x_4(k_3) )}
\newcommand{\matriz}{
      \begin{pmatrix}
	k_1  & -k_1     & 0    & 0 \\
	-k_1 & k_1 +k_2 & -k_2 & 0 \\
	0    & -k_2     & k_2 + k_3 & -k3\\
	0    &  0       & -k_3      & k_3
      \end{pmatrix}
}
\newcommand{\vectorx}{
    \begin{pmatrix}   x_1\\ x_2\\ x_3\\ x_4 \end{pmatrix}
}
\newcommand{\vectorF}{
    \begin{pmatrix} F_1\\ F_2\\ F_3\\ F_4 \end{pmatrix}
}
\newcommand{\solvefa}{
  x_1 k_1 + x_2 (-k_1) + x_3 \cdot (0) +x_4 \cdot (0) &= F_1
}
\newcommand{\solvefb}{
  x_1 \cdot (0) + x_2 \cdot (0) + x_3 \cdot (-k_3) +x_4 \cdot (k_3) &= F_1
}

%%%%%%%%%%%%%%%%%%%%%%%%%%%%%%\tikzsets
%\tikzstyle{every picture}=[remember picture, overlay]
%\tikzstyle{inc}=[fill=red, opacity=0.2]
%\tikzstyle{vin}=[fill=blue, opacity=0.2]
\tikzset{
      inc/.style={fill=red, opacity=0.2},
      vin/.style={fill=blue, opacity=0.2}
}

% defino el template para la diapositiva del título

%%%%%%%%%%%%%%%%%%%%%%%%%%%%%%%%
% Defino la presentación
%%%%%%%%%%%%%%%%%%%%%%%%%%%%%%%%
\title{
  \mode<article>{
\includegraphics[height=1cm]{./PREAMBLE/logo-isabt25.png}
\hfill
\includegraphics[height=1cm]{./PREAMBLE/ISOLOGOCNEA.png}
\hfill
\includegraphics[height=1cm]{./PREAMBLE/logo-unsam.png}
\\} Solucion de Sistemas Lineales Mixtos}
\subtitle[Modelización 2019]{ Modelización de Propiedades y Procesos 2019 }
\author{Ruben Weht\inst{1}\inst{2} \and Mariano Forti\inst{1}\inst{3} }
\institute{
  \inst{1}Instituto de Tecnología Prof. Jorge Sabato
  \and
  \inst{2}Fisica del Sólido, Edificio TANDAR, \url{weht@cnea.gov.ar},
  interno 7104
  \and
  \inst{3}División Aleaciones Especiales, Edificio 47 (microscopía),
  \url{mforti@cnea.gov.ar}, interno 7832
}
\subject{Solucion de Problemas Lineales Mixtos}
\keywords{Elementos Finitos, Sistemas Mixtos}
\date{2020}

% Inicia el documento.
\begin{document}
% Título de la clase. 
\mode<presentation>{
\begin{frame}[plain]
\titlepage
\end{frame}
}
\mode<article>{
  \maketitle
}


\section{Ejemplo: Problema de los Resortes}
\mode<article>


\mode*
\begin{frame}[label=FrameNumeracionGL]
  \frametitle<presentation>{Problema de los Resortes}

  \begin{figure}
    \fbox{
    \includegraphics[width=\textwidth,page=2, trim=5cm 7cm 5cm 8cm, clip=true]
    {./Libreoffice/MEF01_2018.pdf}
  }
  \end{figure}


\end{frame}
\mode<all>
 

\subsection{Fuerzas en los Nodos}
\mode*

\begin{frame}[label=FrameFuerzasElemento1]
  \frametitle<presentation>{Fuerzas sobre el Elemento 1}

  \begin{figure}
    \includegraphics[width=\textwidth,page=3, trim=5cm 8cm 5cm 6cm, clip=true]
    {./Libreoffice/MEF01_2018.pdf}
    \mode<article>{
      \caption{\protect\label{FigureFuerzasElemento1} Fuerzas sobre el elemento 1}
    }
  \end{figure}

  \begin{equation} 
    \label{EqElemento1}
    \begin{split}
      f_1^1 &= -k_1 (x_2 - x_1)\\[10pt]
      f_2^1 &= k_1 (x_2 - x_1)
    \end{split}
    \quad \Rightarrow \quad
     \begin{pmatrix}
       f_1^1\\[10pt]
       f_2^1
     \end{pmatrix}
     =
     \underbrace{
       k_1 
       \begin{pmatrix}
	 1 & -1 \\[10pt]
	 -1 & 1 
       \end{pmatrix}
     }_{ \mathbf{ k_1 ^{el} } }
    \begin{pmatrix}
      x_1 \\[10pt]
      x_2
    \end{pmatrix}
%    
  \end{equation}
          
\end{frame}

\mode<all>


\subsubsection{Fuerzas sobre el Elemento 2}

\mode<article>

Habiendo introducido la notación para
el elemento 1, simplemente tenemos aquí las ecuaciones equivalentes para el
Elemento 2. la diferencia será cuáles son los grados de libertad
involucrados, porque el Elemento 2 ‘conecta’ o ‘acopla’ a los nodos 2 y 3, y por
lo tanto son las posiciones de dichos nodos, $x_2$ y $x_3$ , las que determinarán el
estiramiento del resorte.

Nuevamente, definimos la matriz elemental $\mathbf{k_{el}^2}$

\mode*

\begin{frame}[label=FrameFuerzasElemento2]
  \frametitle<presentation>{Fuerzas sobre el Elemento 2}

  \begin{figure}
    \includegraphics[width=\textwidth,page=4, trim=5cm 8cm 5cm 6cm, clip=true]
    {./Libreoffice/MEF01_2018.pdf}
    \mode<article>{
      \caption{\protect\label{FigureFuerzasElemento2} Fuerzas sobre el elemento 2}
    }
  \end{figure}

  \begin{equation} 
    \label{EqElemento2}
    \begin{split}
      f_1^2 &= -k_2 (x_3 - x_2)\\[10pt]
      f_2^2 &= k_2 (x_3 - x_2)
    \end{split}
    \quad \Rightarrow \quad
     \begin{pmatrix}
       f_1^2\\[10pt]
       f_2^2
     \end{pmatrix}
     =
     \underbrace{
       k_2 
       \begin{pmatrix}
	 1 & -1 \\[10pt]
	 -1 & 1 
       \end{pmatrix}
     }_{ \mathbf{ k_2 ^{el} } }
    \begin{pmatrix}
      x_2 \\[10pt]
      x_3
    \end{pmatrix}
%    
  \end{equation}
          
\end{frame}

\mode<all>


\subsubsection{Fuerzas sobre el elemento 3}

\mode<article>

Por útlimo, para el Elemento 3 el estiramiento queda determinado por los nodos 4 y 3, de manera
que los grados de libertad relevantes son $x_3$ y $x_4$

\mode*


\begin{frame}[label=FrameFuerzasElemento3]
  \frametitle<presentation>{Fuerzas sobre el Elemento 3}

  \begin{figure}
    \includegraphics[width=\textwidth,page=5, trim=5cm 8cm 5cm 6cm, clip=true]
    {./Libreoffice/MEF01_2018.pdf}
    \mode<article>{
      \caption{\protect\label{FigureFuerzasElemento3} Fuerzas sobre el elemento 3}
    }
  \end{figure}

  \begin{equation} 
    \label{EqElemento2}
    \begin{split}
      f_1^3 &= -k_3 (x_4 - x_3)\\[10pt]
      f_2^3 &= k_3 (x_4 - x_3)
    \end{split}
    \quad \Rightarrow \quad
     \begin{pmatrix}
       f_1^2\\[10pt]
       f_2^2
     \end{pmatrix}
     =
     \underbrace{
       k_3 
       \begin{pmatrix}
	 1 & -1 \\[10pt]
	 -1 & 1 
       \end{pmatrix}
     }_{ \mathbf{ k_3 ^{el} } }
    \begin{pmatrix}
      x_3 \\[10pt]
      x_4
    \end{pmatrix}
%    
  \end{equation}
          
\end{frame}

\mode<all>


\subsection{Fuerzas Globales}

\mode<article>

La fuerza total sobre cada uno de los nodos es la suma de todas las fuerzas que actúa sobre cada
uno de ellos. Al escribir estas ecuaciones globales, es posible acoplar a todos los grados de
libertad a partir de las ecuaciones locales, expandiendo cada una de ellas de manera que
aparezcan en forma explícita todos los grados de libertad del problema $x_1$, $x_2$, $x_3$ y
$x_4$. La fuerza global sobre cada nodo puede numerarse $F_1$, $F_2$, $F_3$ y $F_4$ .

\mode*

\begin{frame}[label=FrameFuerzasGlobales]
  \frametitle<presentation>{Fuerzas Globales}

  \begin{figure}
    \fbox{
    \includegraphics[width=\textwidth,page=6, trim=5cm 11cm 3cm 7cm, clip=true]
    {./Libreoffice/MEF01_2018.pdf}
  }
    \mode<article>{
      \caption{\protect\label{FigureFuerzasGlobales} Fuerzas Globales}
    }
  \end{figure}
%
 \tiny 
  \begin{equation}
    \begin{aligned}
      F_1 &= f_1^1  = -k_1( x_2-x_1) =
                   x_1 (k_1) + x_2(-k_1)+ x_3(0) + x_4(0)\\
      F_2 &= f_1^2 + f_2^1 = k_1( x_2-x_1) -k_2(x_3 - x_2) = 
                   x_1 (-k_1) + x_2(k_1+k_2)+ x_3(-k_2) + x_4(0)\\
      F_3 &= f_2^2 + f_1^3 = k_2( x_3-x_2) - k_3 (x_4 - x_3) = 
                    x_1 (0) + x_2(-k_2)+ x_3(k_2 + k_3) + x_4(-k_3)\\
      F_4 &= f_2^3 = k_3( x_4-x_3)=
                     x_1 (0) + x_2(0)+ x_3(-k_3) + x_4(k_3)
    \end{aligned}
  \end{equation}
%
\end{frame}

\mode<article>

Este sistema de ecuaciones puede pensarse en forma Lineal, donde tenemos un
vector de Fuerzas $\mathbf{F}$  y un vector de desplazamientos $\mathbf{x}$. Notar que también
queda definida la matriz de rigidez global delsistema $\mathbf{M}$

\mode*

\begin{frame}[label=FrameMatrizRigidez]
  \frametitle<presentation>{Matriz de Rigidez Global}

  \begin{equation}
    \underbrace{
      \begin{pmatrix}
	k_1  & -k_1     & 0    & 0 \\
	-k_1 & k_1 +k_2 & -k_2 & 0 \\
	0    & -k_2     & k_2 + k_3 & -k3\\
	0    &  0       & -k_3      & k_3
      \end{pmatrix}
    }_{\mathbf{M}}
    \begin{pmatrix}
      x_1\\
      x_2\\
      x_3\\
      x_4
    \end{pmatrix}
    =
    \begin{pmatrix}
      F_1\\
      F_2\\
      F_3\\
      F_4
    \end{pmatrix}
  \end{equation}

\end{frame}

\mode<all>
	

\section{Solucion de sistemas lineales mixtos}

\mode<article>

Recordemos las condiciones de contorno. El nodo 1 se encuentra empotrado por lo
que su desplazamiento está fijo en $x_1=0$. El nodo 4 está obligado a
desplazarse $x_4 = \delta$ de su posición de equilibrio.  La fuerza necesaria
para que estos dos nodos permanezcan en esas condiciones, $F_1$ y $F_4$ son
desconocidas. 

Por otro lado, los nodos 2 y 3 se han desplazado una cantidad desconocida, pero
como suponemos que están en equilibrio estático podemos decir que no hay
fuerzas externas aplicadas en ellos, $F_2 = F_3 = 0$.

En principio, el sistema lineal encontrado representa un sistema de ecuaciones.
Tratemos primero las ecuaciones que corresponden con los desplazamientos
desconocidos $x_2$ y $x_3$ en el ejemplo del resorte. La operatoria consiste en
expandir las ecuaciones correspondientes en todos los grados de libertad, tal y
como aparecen en la ecuación \ref{EqSistemaLinealExpandido}.

Luego deben ser reordenados los términos de los grados de libertad vinculados
$x_1$ y $x_4$ a la derecha del igual, de manera de dejar del lado izquierdo del
igual solo a los términos de los grados de libertad incógnita $x_2$ y $x_3$, 

\mode*

\begin{frame}<presentation>[label=FrameSeparacionSistema]
  \frametitle{Separación del sistema lineal}
  \begin{tikzpicture}
    [      overlay,remember picture
%      inc/.style={fill=red, opacity=0.2},
%      vin/.style={fill=blue, opacity=0.2}
    ]
    \coordinate (O) at (1,0);
    \coordinate (E) at (10,-4);
    \coordinate (L1) at (1,0.5);
    \coordinate (L2) at (3,0.5);
    \coordinate (A) at (1.5,-1 );
    \coordinate (B) at (2.5,-1);
    \coordinate (C) at (5.5, -1);
    % las coordenadas de xvin y xinc
    \coordinate (X4) at (7,-2);
    \coordinate (X1) at (7,-0.5);
    \coordinate (X3) at ($(X4)+(0,0.5)$);
    \coordinate (F3) at (8.5,-1.5);
    \coordinate (F2) at ($(F3)+(0,0.5)$);
    \coordinate (LI) at (1,1);
   % esta es el desplazamiento para el renglon de abajo 
    \coordinate (D) at (0,-0.5); 
    \onslide<2->
    \draw[vin] (X4) rectangle ($(X4)+(L1)$);
    \draw[vin] (X1) rectangle ($(X1)+(L1)$);
    \draw[inc] (X3) rectangle ($(X3)+(LI)$);
    \onslide<2>
    \draw[vin] (A)  rectangle ($(A)+(L1)$);
    \draw[inc] (B)  rectangle ($(B)+(L2)$); 
    \draw[vin] (C)  rectangle ($(C)+(L1)$);
    \draw[vin] (F2) rectangle ($(F2)+(L1)$);
    \onslide<3>
    \draw[vin] ($(D)+(A)$) rectangle ($(A)+(L1)+(D)$);
    \draw[inc] ($(D)+(B)$) rectangle ($(B)+(L2)+(D)$); 
    \draw[vin] ($(D)+(C)$) rectangle ($(C)+(L1)+(D)$);
    \draw[vin] (F3) rectangle ($(F3)+(L1)$);
    \node at (0,0) {};

  \end{tikzpicture} %dejar esta parte junta para que coincidan las posiciones.
  \begin{equation} \label{eqSistemaLineal2}
    \matriz \vectorx  =  \vectorF
  \end{equation}



\onslide<2->{
\begin{equation}\begin{aligned}
    \ecuaciona \\ 
  \onslide<3->{
    \ecuacionb
  }
\end{aligned}\end{equation}
}
\end{frame}

\begin{frame}[label=FrameReordenarEcuaciones23]
  \frametitle<presentation>{Reordenando Ecuaciones Para Incógnitas}

\begin{equation}  \label{EqReordenoIncognitas}
    \begin{aligned}
      \ecuaciona \\  \ecuacionb \\
      \quad \Rightarrow \quad & \\
      \ecuacionao \\ \ecuacionbo
    \end{aligned}
\end{equation}
\onslide<2>
  \mode<presentation>{
  \begin{tikzpicture}[overlay, remember picture]
    %\draw (0,0) grid (11,4);
    \draw[inc] (2.5,0.5) rectangle (6.6,1.5);
    \draw[vin] (7,0.5) rectangle (11,1.5);
  \end{tikzpicture}
}

\end{frame}

\mode<all>

\subsection{Solución de las incógnitas}

\mode<article>
Finalmente, pensemos en un vector de incógnitas $X_r = (x_2, x_3)$   y un
vector de vínculos $X_s  = (x_1, x_4)$ . Con estas definiciones, las
ecuaciones (\ref{EqReordenoIncognitas})  puede escribirse en forma matricial, 

\mode*

  \begin{frame}[label=FrameSistemaReducido]
    \frametitle<presentation>{Reducción del Sistema}
    \begin{equation}\label{EqSistemaReducido}
      \underbrace{
	\begin{pmatrix} k_1 + k_2 & - k_2 \\ -k_2 & k_2 + k_3 \end{pmatrix}
	}_{ \mathbf{K_{red} } }
	\begin{pmatrix}x_2 \\ x_3\end{pmatrix}
	  = 
	  \begin{pmatrix}F_2 \\ F_3\end{pmatrix}
	    -
	  \underbrace{
	      \begin{pmatrix} -k_1 & 0 \\ 0 & -k1 \end{pmatrix}
	   }_{ \mathbf{ K_{vin} } }
	\begin{pmatrix}x_1 \\ x_4\end{pmatrix}
    \end{equation}

    \onslide<2>
    \mode<presentation>{    Resolvemos:}
    \begin{equation} \label{EqSolucionSistema}
      \begin{pmatrix} x_2 \\ x_3  \end{pmatrix}
	=
	\mathbf{K_{red}} ^{-1} 
	\left[
      \begin{pmatrix} F_2 \\ F_3  \end{pmatrix}
	- 
	\mathbf{ K_{vin} }
      \begin{pmatrix} x_1 \\ x_4  \end{pmatrix}
	\right]
    \end{equation}

  \end{frame}

  \mode<article>

de manera que quedarán definidas la \textbf{matriz de rigidez reducida} $\mathbf{K_{red} }$ y la
\textbf{matriz de coeficientes vínculos} $\mathbf{K_{vin}}$. La solución del sistema puede
hallarse en forma inmediata como se muestra en la ecuación (\ref{EqSolucionSistema}).

\mode*

\mode<all>


\mode<article>

El problema no está terminado aún puesto que falta resolver las fuerzas $F_1$ y
$F_4$. Pero, conociendo los desplazamientos a partir de la ecuación
\ref{EqSolucionSistema}, resulta trivial resolver a partír de las ecuaciones
correspondientes en el sistema de ecuaciones \ref{EqSistemaReducido}. Aún así,
por completitud, escribamos la forma matricial de las ecuaciones
correspondientes. Tomemos un vector de fuerzas de vínculo $F_s = (F_1, F_4)$,
el sistema lineal equivalente al subconjunto de ecuaciones correspondientes es 

\mode*
\begin{frame}[label=FrameSolveForces]
  \frametitle<presentation>{Solución de las fuerzas dev vínculo}
  \mode<presentation>{
  \begin{equation}    \begin{aligned}
      \matriz \vectorx = \vectorF
    \end{aligned}  \end{equation}
  \begin{tikzpicture}[overlay, remember picture]
    \draw (0,0) grid (10,2);
    \coordinate (LX) at (5,0.5);
    \coordinate (A)  at (1.5,1.7);
    \coordinate (B)  at (1.5,0.3);
    \coordinate (C)  at (6.8, 0) ;
    \coordinate (LY) at (1,2.3) ;
    \coordinate (F4) at (8.5, 0.3);
    \coordinate (F1) at (8.5, 1.8);
    \coordinate (LF) at(0.5, 0.5);
    \draw[vin]  (A) rectangle ($(A)+(LX)$);
    \draw[vin]  (B) rectangle ($(B)+(LX)$);
    \draw[vin]  (C) rectangle ($(C) + (LY)$);
    \draw[vin]  (F1) rectangle ($(F1)+(LF)$);
    \draw[vin]  (F4) rectangle ($(F4)+(LF)$);
  \end{tikzpicture}
    \onslide<2->
  \begin{equation}
    \begin{aligned}
      \solvefa \\
      \solvefb
    \end{aligned}
  \end{equation}
}
  \onslide<3>

    \begin{equation}\label{EqSolucionFuerzasVinc}
      \begin{pmatrix}F_1 \\ F_4 \end{pmatrix} 
	= 
	\begin{pmatrix} k_1  & -k_1 & 0 & 0 \\ 0 & 0 & -k_3 & k_3 \end{pmatrix} 
      \vectorx
    \end{equation}


\end{frame}

\mode<all>


\end{document}
