% modo presentación. si se cambia a modo notas se imprimen las notas (creo)
%\documentclass[xcolor={dvipsnames,x11names,svgnames},aspectratio=169,notes]{beamer}
% \mode<presentation>
%handout de las diapositivas, comentar arriva!
% handout de la presentacion
%\documentclass[handout,xcolor={dvipsnames,x11names,svgnames},aspectratio=169,notes]{beamer}
%\mode<handout>

% paquete beamer, opciones para nombres de colores y aspect ratio de panatalla grande.
%\documentclass[a4paper,12pt]{article}
%\usepackage{beamerarticle}

% para no tener problemas con acentos etc.
\usepackage[utf8]{inputenc}
% en español
\usepackage[spanish]{babel}
%matemática
\usepackage{amsmath}
% este no se si hace falta pero por las dudas
\usepackage{graphicx}
% para incluir peliculas
\usepackage{multimedia}
% para usar segunda pantalla
\usepackage{pgfpages}
\usepackage{pgf}
% para hacer dibujitos
\usepackage{tikz}
\usetikzlibrary[automata,calc,arrows,decorations.pathmorphing,backgrounds,shapes,
patterns,positioning,fit,petri,overlay-beamer-styles]
\tikzstyle{every picture}+=[remember picture]
%recuadros sencillos
%\usepackage{tcolorbox}
% enumeradores intercambiables
\usepackage{enumerate}
% para subtitulos en figuras
\usepackage{subcaption}
%listings for code input
\usepackage{listings}
%verbatim input from file!
\usepackage{verbatim}
\usepackage{fancyvrb}
% para modificar los encabezados y pies de página.
%\usepackage{fancyhdr}
%\pagestyle{fancy}
\usepackage{standalone}

\definecolor{codegreen}{rgb}{0,0.6,0}
\definecolor{codegray}{rgb}{0.5,0.5,0.5}
\definecolor{codepurple}{rgb}{0.58,0,0.82}
\definecolor{backcolour}{rgb}{0.95,0.95,0.92}

\lstdefinestyle{codeblock}{
    backgroundcolor=\color{backcolour},   
    commentstyle=\color{codegreen},
    keywordstyle=\color{magenta},
    numberstyle=\tiny\color{codegray},
    stringstyle=\color{codepurple},
    basicstyle=\ttfamily\footnotesize,
    breakatwhitespace=false,         
    breaklines=true,                 
    captionpos=b,                    
    keepspaces=true,                 
    numbers=left,                    
    numbersep=5pt,                  
    showspaces=false,                
    showstringspaces=false,
    showtabs=false,                  
    tabsize=2
}


% para mostrar las notas en modo presentacion. 
%\setbeameroption{show notes}
%para ocultar las notas
%\setbeameroption{hide notes}
%para dejar las notas en la segunda patnalla
%\setbeameroption{show notes on second screen}
%incluyo los paquetes necesarios

% en caso de handouts, ver 2 en 1 o 4 en 


% en forma arbitraria decid que los parrafos no llevan indentación.
\setlength{\parindent}{0cm}

% incluyo los beamercolors
\include{./PREAMBLE/BEAMERCOLORS}

%incluyo el tema y modificaciones
\include{./PREAMBLE/BEAMERTHEME}

% modifico los temas
\include{./PREAMBLE/THEMES}

% defino el template para la diapositiva del título

%%%%%%%%%%%%%%%%%%%%%%%%%%%%%%%%
% Defino la presentación
%%%%%%%%%%%%%%%%%%%%%%%%%%%%%%%%
\title{
  \mode<article>{
\includegraphics[height=1cm]{./PREAMBLE/logo-isabt25.png}
\hfill
\includegraphics[height=1cm]{./PREAMBLE/ISOLOGOCNEA.png}
\hfill
\includegraphics[height=1cm]{./PREAMBLE/logo-unsam.png}
\\} Solucion de Sistemas Lineales Mixtos}
\subtitle[Modelización 2019]{ Modelización de Propiedades y Procesos 2019 }
\author{Ruben Weht\inst{1}\inst{2} \and Mariano Forti\inst{1}\inst{3} }
\institute{
  \inst{1}Instituto de Tecnología Prof. Jorge Sabato
  \and
  \inst{2}Fisica del Sólido, Edificio TANDAR, \url{weht@cnea.gov.ar},
  interno 7104
  \and
  \inst{3}División Aleaciones Especiales, Edificio 47 (microscopía),
  \url{mforti@cnea.gov.ar}, interno 7832
}
\subject{Solucion de Problemas Lineales Mixtos}
\keywords{Elementos Finitos, Sistemas Mixtos}
\date{2020}

% Inicia el documento.
\begin{document}
% Título de la clase. 
\mode<presentation>{
\begin{frame}[plain]
\titlepage
\end{frame}
}
\mode<article>{
  \maketitle
}

\section{Ejemplo: Problema de los Resortes}
\mode<article>


\mode*
\begin{frame}[label=FrameNumeracionGL]
  \frametitle<presentation>{Problema de los Resortes}

  \begin{figure}
    \fbox{
    \includegraphics[width=\textwidth,page=2, trim=5cm 7cm 5cm 8cm, clip=true]
    {./Libreoffice/MEF01_2018.pdf}
  }
  \end{figure}


\end{frame}
\mode<all>
 

\subsection{Fuerzas en los Nodos}
\mode*

\begin{frame}[label=FrameFuerzasElemento1]
  \frametitle<presentation>{Fuerzas sobre el Elemento 1}

  \begin{figure}
    \includegraphics[width=\textwidth,page=3, trim=5cm 8cm 5cm 6cm, clip=true]
    {./Libreoffice/MEF01_2018.pdf}
    \mode<article>{
      \caption{\protect\label{FigureFuerzasElemento1} Fuerzas sobre el elemento 1}
    }
  \end{figure}

  \begin{equation} 
    \label{EqElemento1}
    \begin{split}
      f_1^1 &= -k_1 (x_2 - x_1)\\[10pt]
      f_2^1 &= k_1 (x_2 - x_1)
    \end{split}
    \quad \Rightarrow \quad
     \begin{pmatrix}
       f_1^1\\[10pt]
       f_2^1
     \end{pmatrix}
     =
     \underbrace{
       k_1 
       \begin{pmatrix}
	 1 & -1 \\[10pt]
	 -1 & 1 
       \end{pmatrix}
     }_{ \mathbf{ k_1 ^{el} } }
    \begin{pmatrix}
      x_1 \\[10pt]
      x_2
    \end{pmatrix}
%    
  \end{equation}
          
\end{frame}

\mode<all>


\subsubsection{Fuerzas sobre el Elemento 2}

\mode<article>

Habiendo introducido la notación para
el elemento 1, simplemente tenemos aquí las ecuaciones equivalentes para el
Elemento 2. la diferencia será cuáles son los grados de libertad
involucrados, porque el Elemento 2 ‘conecta’ o ‘acopla’ a los nodos 2 y 3, y por
lo tanto son las posiciones de dichos nodos, $x_2$ y $x_3$ , las que determinarán el
estiramiento del resorte.

Nuevamente, definimos la matriz elemental $\mathbf{k_{el}^2}$

\mode*

\begin{frame}[label=FrameFuerzasElemento2]
  \frametitle<presentation>{Fuerzas sobre el Elemento 2}

  \begin{figure}
    \includegraphics[width=\textwidth,page=4, trim=5cm 8cm 5cm 6cm, clip=true]
    {./Libreoffice/MEF01_2018.pdf}
    \mode<article>{
      \caption{\protect\label{FigureFuerzasElemento2} Fuerzas sobre el elemento 2}
    }
  \end{figure}

  \begin{equation} 
    \label{EqElemento2}
    \begin{split}
      f_1^2 &= -k_2 (x_3 - x_2)\\[10pt]
      f_2^2 &= k_2 (x_3 - x_2)
    \end{split}
    \quad \Rightarrow \quad
     \begin{pmatrix}
       f_1^2\\[10pt]
       f_2^2
     \end{pmatrix}
     =
     \underbrace{
       k_2 
       \begin{pmatrix}
	 1 & -1 \\[10pt]
	 -1 & 1 
       \end{pmatrix}
     }_{ \mathbf{ k_2 ^{el} } }
    \begin{pmatrix}
      x_2 \\[10pt]
      x_3
    \end{pmatrix}
%    
  \end{equation}
          
\end{frame}

\mode<all>


\end{document}
